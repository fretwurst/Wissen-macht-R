\documentclass[twoside, pagesize, fontsize=11pt, dvipsnames]{scrreport}

\usepackage[ngerman]{babel}
\usepackage{anyfontsize} % jede Schriftgrösse eben
\usepackage[table]{xcolor}
\usepackage{colortbl}
\usepackage{tabu}
\usepackage{svg}
\usepackage{fontenc}
\usepackage{eurosym}
\usepackage{soul}           %% Sperren und Schafe stehlen

 \usepackage{enumerate}      %% Nummerieren
 \usepackage{ifthen}         %% Wenn-Dann eben
 \usepackage{array}          %% besserer Mathesatz
 \usepackage[fleqn,reqno]{amsmath}            %% für Mathematiksatz
 \usepackage{amssymb}           %% für Symbole

 \usepackage{marginnote}        %% Randnotizen (für Hypref)
 \usepackage{booktabs}          %% bessere Linien in Tabellen
 \usepackage{longtable}
 \usepackage{makecell}
\usepackage{graphicx}
 \usepackage{float}             %% Gleitobjektumgebungen
 \usepackage{flafter}           %% floats hinter ersten Verweis
 \usepackage{rotfloat, rotating}      %% drehen
 \usepackage{morefloats}    %% erweitert die Möglichkeiten

\usepackage[most]{tcolorbox}

 \usepackage{titletoc}
 \usepackage{afterpage}

 \usepackage{hanging}
 \usepackage{natbib}
%\bibliographystyle{apalike}
% \usepackage{babelbib}
% \usepackage{bibgerm}

%% für huxtables
\usepackage{calc}
%\usepackage{tabularx}
\usepackage{threeparttable}
% ende für huxtables
\usepackage{wrapfig}
 \usepackage{fancybox}          %% Schöne Boxen
 \usepackage{nameref}
 \usepackage[ngerman]{varioref}

 \usepackage{pdfpages}
 \usepackage{setspace}
 \usepackage{boxedminipage}     %% umrahmte Boxen
 \usepackage{multicol}          %% mehrspaltige Zeilen
 \usepackage{epsfig,lscape}     %% Graphiken auch Quer

\usepackage[stable, norule, flushmargin]{footmisc} %% Fußgut

 \usepackage[np]{numprint}  %%Zahlen an Komma ausrichten

 \usepackage[mla]{ellipsis}       %% Auslassungspunkte
% \usepackage[newcommands]{ragged2e} %Verbesserung der Seitenaufteilung
 \usepackage[safe]{textcomp} %zusätzliche Zeichen, wie schwarze Punkte

%%%% hyperref %%%%
\usepackage[hyphens]{url} %% URLs sauber einfügen und umbrechen - muss vor hyperref
\usepackage[pdfa=true, pdflang = de, colorlinks = true, allcolors = darkgray]{hyperref}
%%%% muss alles hinter hyperref, weil es sonst package-clashes erzeugt:
\usepackage{accsupp}    %% für Barrierefreiheit
\usepackage{pdfcomment} %% für Barrierefreiheit
%\PassOptionsToPackage{hyphens}{accsupp} %% übergibt hyphens für das url-Paket, das in accsupp ohne diese Option geladen wird.
\PassOptionsToPackage{hyphens}{pdfcomment} %% übergibt hyphens für das url-Paket, das in pdfcomment ohne diese Option geladen wird.
%%%%

\usepackage{twoopt}

\usepackage{paralist}
\usepackage{siunitx}

\usepackage{pdflscape}


%\usepackage[style=authoryear,maxcitenames=2, backend=biber, isbn=false,doi=false, eprint = false]{biblatex}

%___________________  Pakete Ende  ____________________________________________


\usepackage{color}
\usepackage{fancyvrb}
\newcommand{\VerbBar}{|}
\newcommand{\VERB}{\Verb[commandchars=\\\{\}]}
\DefineVerbatimEnvironment{Highlighting}{Verbatim}{commandchars=\\\{\}}
% Add ',fontsize=\small' for more characters per line
\usepackage{framed}
\definecolor{shadecolor}{RGB}{241,243,245}
\newenvironment{Shaded}{\begin{snugshade}}{\end{snugshade}}
\newcommand{\AlertTok}[1]{\textcolor[rgb]{0.68,0.00,0.00}{#1}}
\newcommand{\AnnotationTok}[1]{\textcolor[rgb]{0.37,0.37,0.37}{#1}}
\newcommand{\AttributeTok}[1]{\textcolor[rgb]{0.40,0.45,0.13}{#1}}
\newcommand{\BaseNTok}[1]{\textcolor[rgb]{0.68,0.00,0.00}{#1}}
\newcommand{\BuiltInTok}[1]{\textcolor[rgb]{0.00,0.23,0.31}{#1}}
\newcommand{\CharTok}[1]{\textcolor[rgb]{0.13,0.47,0.30}{#1}}
\newcommand{\CommentTok}[1]{\textcolor[rgb]{0.37,0.37,0.37}{#1}}
\newcommand{\CommentVarTok}[1]{\textcolor[rgb]{0.37,0.37,0.37}{\textit{#1}}}
\newcommand{\ConstantTok}[1]{\textcolor[rgb]{0.56,0.35,0.01}{#1}}
\newcommand{\ControlFlowTok}[1]{\textcolor[rgb]{0.00,0.23,0.31}{#1}}
\newcommand{\DataTypeTok}[1]{\textcolor[rgb]{0.68,0.00,0.00}{#1}}
\newcommand{\DecValTok}[1]{\textcolor[rgb]{0.68,0.00,0.00}{#1}}
\newcommand{\DocumentationTok}[1]{\textcolor[rgb]{0.37,0.37,0.37}{\textit{#1}}}
\newcommand{\ErrorTok}[1]{\textcolor[rgb]{0.68,0.00,0.00}{#1}}
\newcommand{\ExtensionTok}[1]{\textcolor[rgb]{0.00,0.23,0.31}{#1}}
\newcommand{\FloatTok}[1]{\textcolor[rgb]{0.68,0.00,0.00}{#1}}
\newcommand{\FunctionTok}[1]{\textcolor[rgb]{0.28,0.35,0.67}{#1}}
\newcommand{\ImportTok}[1]{\textcolor[rgb]{0.00,0.46,0.62}{#1}}
\newcommand{\InformationTok}[1]{\textcolor[rgb]{0.37,0.37,0.37}{#1}}
\newcommand{\KeywordTok}[1]{\textcolor[rgb]{0.00,0.23,0.31}{#1}}
\newcommand{\NormalTok}[1]{\textcolor[rgb]{0.00,0.23,0.31}{#1}}
\newcommand{\OperatorTok}[1]{\textcolor[rgb]{0.37,0.37,0.37}{#1}}
\newcommand{\OtherTok}[1]{\textcolor[rgb]{0.00,0.23,0.31}{#1}}
\newcommand{\PreprocessorTok}[1]{\textcolor[rgb]{0.68,0.00,0.00}{#1}}
\newcommand{\RegionMarkerTok}[1]{\textcolor[rgb]{0.00,0.23,0.31}{#1}}
\newcommand{\SpecialCharTok}[1]{\textcolor[rgb]{0.37,0.37,0.37}{#1}}
\newcommand{\SpecialStringTok}[1]{\textcolor[rgb]{0.13,0.47,0.30}{#1}}
\newcommand{\StringTok}[1]{\textcolor[rgb]{0.13,0.47,0.30}{#1}}
\newcommand{\VariableTok}[1]{\textcolor[rgb]{0.07,0.07,0.07}{#1}}
\newcommand{\VerbatimStringTok}[1]{\textcolor[rgb]{0.13,0.47,0.30}{#1}}
\newcommand{\WarningTok}[1]{\textcolor[rgb]{0.37,0.37,0.37}{\textit{#1}}}

%\def\tightlist{}

%%%%%%%%%%%%%%%%%%%%%  Größen ändern gegen LaTeXs Hysterie %%%%%%%%%%%%%%%%%%%%%%%%%%%%%%
\widowpenalty=10000 % erschwert Hurenkindern das Leben
\clubpenalty=3000 % erschwert Schusterjungen das Leben
\tolerance=300  %Wortabstände dehnbarer; fürs Final auf 300!!
\hbadness= 300
  \emergencystretch= 1.5em
 \hfuzz = 0.3pt
 \vfuzz \hfuzz
 \raggedbottom

%% erste Zeile einrücken
\setlength{\parindent}{1em}
\newcommand{\Absatz}{\vspace{4ex} \noindent}

%% Einzug für Formeln
\setlength\mathindent{1pc}
%% Fußnoten
%\counterwithout{footnote}{chapter} % über Kapitel hinweg zählen
\addtolength{\skip\footins}{1ex plus 2mm} % Abstand zum Text
\renewcommand{\footnotesep}{1ex} % da nummeriert, Abstand ok
% rechtsbündig aus likem Rand

\let\Umathcode\XeTeXmathcode \let\Umathchardef\XeTeXmathchardef

%____________ Größen Ende ______________________________

%%%%% Farbboxen für Hinweise ------
\tcbset{textmarker/.style={%
        enhanced,
        parbox=false,boxrule=0mm,boxsep=0mm,arc=0mm,
        outer arc=0mm,left=6mm,right=3mm,top=7pt,bottom=7pt,
        toptitle=1mm,bottomtitle=1mm,oversize}}

\newtcolorbox{hintBox}{textmarker,
    borderline west={6pt}{0pt}{yellow},
    colback=yellow!10!white}

\newtcolorbox{importantBox}{textmarker,
    borderline west={6pt}{0pt}{red},
    colback=red!10!white}

\newtcolorbox{noteBox}{textmarker,
    borderline west={6pt}{0pt}{green},
    colback=green!10!white}

\newcommand{\Anmerkung}[1]
{\begin{hintBox}
#1
\end{hintBox}}

\newcommand{\Alarm}[1]
{\begin{importantBox}
#1
\end{importantBox}}

\newcommand{\Allesgut}[1]
{\begin{noteBox}
#1
\end{noteBox}}


\newcommand{\on}[1]{{#1}} %deaktiviertes oldstylenums
\newcommand{\p}{{\npnoaddmissingzero \npdecimalsign{.}}} %p-Wert
                                %kompakt in zB Kovarianztabellen

%%%%%%%%%      Umgebungen und Kommandos   %%%%%%%%%%%%%%%%%%%%%%%%%%%%

%%%%  Boxen

\newtcolorbox{IYI_C}{
  colback=white,
  colframe=MidnightBlue,
  coltext=black,
  boxsep=5pt,
  arc=4pt}

\newenvironment{IYI}[1]
  {
  \begin{itemize}
  \renewcommand{\labelitemi}{
    \raisebox{-.9\height}[0pt][0pt]{
      {\setkeys{Gin}{width= 3em,keepaspectratio}
        \hspace*{-3cm}\includegraphics{images/IYI.pdf}}
    }
  }
  \setlength{\fboxsep}{1em}
  \begin{IYI_C}
  \item
  }
  {
  \end{IYI_C}
  \end{itemize}
  }

\newtcolorbox{QA_C}{
  colback=white,
  colframe=orange,
  coltext=black,
  boxsep=5pt,
  arc=4pt}

  \newenvironment{QA}[1]
  {
  \begin{itemize}
  \renewcommand{\labelitemi}{
    \raisebox{-.9\height}[0pt][0pt]{
      {\setkeys{Gin}{width=3em,keepaspectratio}
        \hspace{-2cm}\includegraphics{images/QA.pdf}}
    }
  }
  \setlength{\fboxsep}{1em}
  \begin{QA_C}
  \item
  }
  {
  \end{QA_C}
  \end{itemize}
  }

%%% ENDE Boxen

%% \PBS für geschützten Backslash
\newcommand{\PreserveBackslash}[1]{\let\temp=\\#1\let\\=\temp}
\let\PBS=\PreserveBackslash

\renewcommand{\labelitemi}{--}

%%% Bilder einbauen (breitbild = Seitenbreite skalieren; bild zentriert
\newcommandtwoopt{\Bild}[3][1][0 0 0
0]{ \begin{center}
    \includegraphics[width=#1\linewidth,trim = #2, clip]{images/#3}
  \end{center}}
%%%% Aufzählungen

\renewcommand{\labelitemi}{\small\textbullet}

\newcommand{\Beschrieb}[1]{\ifthenelse{\isundefined{#1}}{}{
\begin{minipage}[H]{1.0\linewidth}
  \begin{multicols}{3}
 #1
 \end{multicols}
\end{minipage}\vspace{5mm}}}

%% für Barrierefreiheit
\newcommand{\AccTool}[2]{\BeginAccSupp{method=pdfstringdef,unicode,Alt={{#1}}}\pdftooltip{{#2}}{{#1}}\EndAccSupp{}}

%__________________ Umgebungen Ende _____________________________________________

%--------  Tabellenbefehle  ----------------------
%LaTeX interpretiert Tabulatoren und Zeilenenden für Tabellen
% & und \\ können eingesetzt werden, wenn nötig
% bei zB \\\midrule%! muss das Zeilenende auskommentiert sein

\renewcommand{\floatpagefraction}{.7}
\renewcommand{\textfraction}{.12}
\renewcommand{\topfraction}{.8}     % vorher: .7
\renewcommand{\bottomfraction}{.5}  % vorher: .3
\makeatletter
\renewcommand{\fps@figure}{htbp}
\renewcommand{\fps@table}{htbp}
\makeatother

\setbox0=\hbox{%
  \begin{tabular}{c}
 \global\let\CsvNewline\\%
 \end{tabular}}
 {\catcode`\^^M=\active%
  \gdef\CsvObeylines{\catcode`\^^M=\active \let^^M=\CsvNewline}}%

\setcapindent{1em}

\newcommand{\TBZeilenabstand}{\aboverulesep = 0pt \belowrulesep = 0pt}

\newcommand{\NormalZeilenabstand}{\aboverulesep = 0.605mm \belowrulesep = 0.984mm}


% \newcommand{\corcmidrule}[1][2pt]{% \corcmidrule[<len>]
%   \\[\dimexpr-\normalbaselineskip-\belowrulesep-\aboverulesep-#1\relax]%
% }

% Correct for \cmidrule colour adjustment/vertical skip
\newcommand{\corcmidrule}[1][2pt]{% \corcmidrule[<len>]
  \\[\dimexpr-\arraystretch\normalbaselineskip-\belowrulesep-\aboverulesep-#1\relax]%
}

 \newcommand{\cgrauDurch}{\arrayrulecolor{black!30}\specialrule{.2pt}{0pt}{0pt}\arrayrulecolor{black}}
 \newcommand{\cgrau}[1]{\arrayrulecolor{black!30} \cmidrule[.8pt](l{3pt}r{2pt}){#1} \corcmidrule[1pt] \arrayrulecolor{black}}

%% variable Verweise
\renewcommand\reftextfaceafter{gegen\"uberliegend}
\renewcommand\reftextfacebefore{gegen\"uberliegend}
\renewcommand\reftextbefore{vorherige Seite}
\renewcommand\reftextafter{n\"achste Seite}
\renewcommand\reftextcurrent{}
\renewcommand\reftextfaraway[1]{Seite~\pageref{#1}}

\makeatletter
\@addtoreset{figure}{section}
\@addtoreset{table}{section}
\makeatother

\renewcommand{\thefigure}{\thesection.\arabic{figure}}
\renewcommand{\thetable}{\thesection.\arabic{table}}

%\renewcommand{\bminipage}{\begin{minipage}[t]{.9\textwidth}}
%\renewcommand{\eminipage}{\end{minipage}}

%%%%%%%%%%%%%%%%%%%%%%%%%%%%%%%%%%%%%%%%%%%%%%%%
%    aus Bookdown
%%%%%%%%%%%%%%%%%%%%%%%%%%%%%%%%%%%%%%%%%%%%%%%%
%
\definecolor{shadecolor}{RGB}{248,248,248}
% \newenvironment{Shaded}{\begin{snugshade}}{\end{snugshade}}
%\newcommand{\AlertTok}[1]{\textcolor[rgb]{0.94,0.16,0.16}{#1}}
% \newcommand{\AnnotationTok}[1]{\textcolor[rgb]{0.56,0.35,0.01}{\textbf{\textit{#1}}}}
% \newcommand{\AttributeTok}[1]{\textcolor[rgb]{0.77,0.63,0.00}{#1}}
% \newcommand{\BaseNTok}[1]{\textcolor[rgb]{0.00,0.00,0.81}{#1}}
% \newcommand{\BuiltInTok}[1]{#1}
% \newcommand{\CharTok}[1]{\textcolor[rgb]{0.31,0.60,0.02}{#1}}
% \newcommand{\CommentTok}[1]{\textcolor[rgb]{0.56,0.35,0.01}{\textit{#1}}}
% \newcommand{\CommentVarTok}[1]{\textcolor[rgb]{0.56,0.35,0.01}{\textbf{\textit{#1}}}}
% \newcommand{\ConstantTok}[1]{\textcolor[rgb]{0.00,0.00,0.00}{#1}}
% \newcommand{\ControlFlowTok}[1]{\textcolor[rgb]{0.13,0.29,0.53}{\textbf{#1}}}
% \newcommand{\DataTypeTok}[1]{\textcolor[rgb]{0.13,0.29,0.53}{#1}}
% \newcommand{\DecValTok}[1]{\textcolor[rgb]{0.00,0.00,0.81}{#1}}
% \newcommand{\DocumentationTok}[1]{\textcolor[rgb]{0.56,0.35,0.01}{\textbf{\textit{#1}}}}
% \newcommand{\ErrorTok}[1]{\textcolor[rgb]{0.64,0.00,0.00}{\textbf{#1}}}
% \newcommand{\ExtensionTok}[1]{#1}
% \newcommand{\FloatTok}[1]{\textcolor[rgb]{0.00,0.00,0.81}{#1}}
% \newcommand{\FunctionTok}[1]{\textcolor[rgb]{0.00,0.00,0.00}{#1}}
% \newcommand{\ImportTok}[1]{#1}
% \newcommand{\InformationTok}[1]{\textcolor[rgb]{0.56,0.35,0.01}{\textbf{\textit{#1}}}}
% \newcommand{\KeywordTok}[1]{\textcolor[rgb]{0.13,0.29,0.53}{\textbf{#1}}}
% \newcommand{\NormalTok}[1]{#1}
% \newcommand{\OperatorTok}[1]{\textcolor[rgb]{0.81,0.36,0.00}{\textbf{#1}}}
% \newcommand{\OtherTok}[1]{\textcolor[rgb]{0.56,0.35,0.01}{#1}}
% \newcommand{\PreprocessorTok}[1]{\textcolor[rgb]{0.56,0.35,0.01}{\textit{#1}}}
% \newcommand{\RegionMarkerTok}[1]{#1}
% \newcommand{\SpecialCharTok}[1]{\textcolor[rgb]{0.00,0.00,0.00}{#1}}
% \newcommand{\SpecialStringTok}[1]{\textcolor[rgb]{0.31,0.60,0.02}{#1}}
% \newcommand{\StringTok}[1]{\textcolor[rgb]{0.31,0.60,0.02}{#1}}
% \newcommand{\VariableTok}[1]{\textcolor[rgb]{0.00,0.00,0.00}{#1}}
% \newcommand{\VerbatimStringTok}[1]{\textcolor[rgb]{0.31,0.60,0.02}{#1}}
% \newcommand{\WarningTok}[1]{\textcolor[rgb]{0.56,0.35,0.01}{\textbf{\textit{#1}}}}

\makeatletter
\def\maxwidth{\ifdim\Gin@nat@width>\linewidth\linewidth\else\Gin@nat@width\fi}
\def\maxheight{\ifdim\Gin@nat@height>\textheight\textheight\else\Gin@nat@height\fi}
\makeatother

\providecommand{\tightlist}{%
  \setlength{\itemsep}{0pt}\setlength{\parskip}{0pt}}
\setcounter{secnumdepth}{5}

\ifx\paragraph\undefined\else
\let\oldparagraph\paragraph
\renewcommand{\paragraph}[1]{\oldparagraph{#1}\mbox{}}
\fi
\ifx\subparagraph\undefined\else
\let\oldsubparagraph\subparagraph
\renewcommand{\subparagraph}[1]{\oldsubparagraph{#1}\mbox{}}
\fi

%%% Use protect on footnotes to avoid problems with footnotes in titles
\let\rmarkdownfootnote\footnote%
\def\footnote{\protect\rmarkdownfootnote}

%%% Change title format to be more compact
\usepackage{titling}

%%%%%%%%%%%%%%%%%%%%%%%%%%%% aus bookdown Ende %%%%%%%%%%%%%%%%%%%%%%%%%%%%%%%%%%%%%%%%%%%


\hypersetup { pdfa=true, pdflang = de,colorlinks = true, linkcolor =
  black, urlcolor = black, pdfauthor={Dr. Fretwurst, Benjamin},
  pdfsubject={Statistik Aufbau}, pdftitle={Wissen macht R!},
  pdfkeywords={Statistik, lineares Modell, Regression, Varianzanalyse, Clusteranalyse} lang = {DE} }
\urlstyle{same}

%\addbibresource{../PeM.bib}
%\ExecuteBibliographyOptions{url=true, doi = false}

\usepackage[a4paper, left=30mm, asymmetric, top=30mm,textwidth=15cm,
textheight=22.5cm]{geometry}

% Headers and footers
\usepackage{scrlayer-scrpage}
\pagestyle{scrheadings}

  \clearscrheadfoot
 \automark[section]{section}
  \ihead[]{\headmark} %oben links
  \chead[]{} %oben Mitte
  \ohead{\pagemark} %oben rechts Logo über AddToShipoutPicture unten
  \ifoot[]{\footnotesize{}} %unten links
  \cfoot[]{} %unten Mitte
  \ofoot[]{\footnotesize{}} % unten rechts
  \setheadsepline{.2pt}
  \setfootsepline{.2pt}
\addtokomafont{disposition}{\sffamily}

\addtokomafont{disposition}{\sffamily}
%\renewcommand{\familydefault}{\sfdefault}

\usepackage[labelsep=colon]{caption}
% \captionsetup[table]{textfont = {sf}, labelfont = {sf}} %bf
%\renewcaptionname{ngerman}{\tablename}{Tabelle}

%\usepackage{baskervald} % auch nicht schlecht
\usepackage[]{CrimsonPro} % gut
%\setmainfont{Century Gothic} %schlimm, aber IaKom und darum schön

\usepackage[german=swiss]{csquotes}

%%%%%%%%%%%%%%%%%%%%%%%%%%%%%%%%%%%%%%%%%%%%

              %Dokumentstart

%%%%%%%%%%%%%%%%%%%%%%%%%%%%%%%%%%%%%%%%%%%%

\hyphenation{
al-ler-dings
Al-ters-grup-pen
Be-reich
Co-ro-na-kri-se
Fern-seh-an-ge-bo-te
Fern-seh-ge-rät
Fern-seh-ge-rä-te
Fest-netz-tele-fo-nie
Ge-samt-zu-frie-den-heit
glei-chen
glei-chen
In-for-ma-tions-suche
in-nen
ita-li-e-nische
je-weils
po-li-ti-schem
po-li-ti-schen
si-tu-a-ti-ve
si-tu-a-tiv
Sta-bi-li-tät
Ver-schwö-rungs-theo-re-ti-ker
Ver-trau-ens-kluft
Zeit-um-fänge
Zu-schau-er
zweit-höchs-ten
}

\begin{document}

\begin{titlepage}
\sffamily
\setlength\parindent{0pt}

\hfill \includegraphics[width = 6cm]{images/uzh_logo_d_pos.pdf}\par

\vspace{5cm}


{\bfseries \fontsize{20}{26} \selectfont Wissen macht R!} \\[1ex]


\vfill

Dr. Benjamin Fretwurst\\

\vfill
{\large }
\raggedright



12/12/22 \\[.5cm]
%\textbf{[\today] \\[1.5cm]}
\pagenumbering{Roman}
\clearpage

\vspace*{4cm}

Quellenangabe\\[.5cm]



%\end{tabbing}
\end{titlepage}


\makeatletter
\AddToShipoutPicture{\setlength{\unitlength}{1cm}\put(24.32,18.65){{\includegraphics[height=.6cm]{images/uzh_logo_d_pos.pdf}}}}
\makeatletter

%Überschriften unterdrücken durch IV und TV (wegen multicolumn)
\renewcommand{\listoftables}{\@starttoc{lot}}
\renewcommand{\tableofcontents}{\@starttoc{toc}}
\renewcommand{\listoffigures}{\@starttoc{lof}}

%Mehr Platz für breite Tabellennummern
\renewcommand{\l@table}{\@dottedtocline{1}{1em}{3em}}
\makeatother

\clearpage
\section*{Inhalt}
\label{sec:inhalt}
\pdfbookmark[1]{\contentsname}{toc}
%  \begin{multicols}{2}

\tableofcontents
%  \end{multicols}

\clearpage

\section*{Tabellen- und Abbildungsverzeichnis}

\pdfbookmark[1]{Tabellen/Abbildungen}{lot}

\section*{Abbildungsverzeichnis}
\label{sec:Abbildungsverzeichnis}

%\begin{multicols}{2}
\listoffigures
%\end{multicols}

\section*{Tabellenverzeichnis}
\label{sec:tabellenverzeichnis}

%\begin{multicols}{2}
\listoftables
%\end{multicols}

\cleardoublepage

\pagenumbering{arabic}
%%%%%%%%%%%%%%%%%%%%%%%%%%%%%%%%%%

             %Body

%%%%%%%%%%%%%%%%%%%%%%%%%%%%%%%%%%

\bookmarksetup{startatroot}

\hypertarget{intro}{%
\chapter{Intro}\label{intro}}

R ist ein kostenloses, erweiterbares, modernes und freundliches
Statistiksystem (programmierbar mit Lösungen in stapelweise Paketen).
Mit R-Studio können Sie Ihre Datenverarbeitung und -analysen in R für
sich und andere sehr gut dokumentieren. Mit den Einführungen auf dieser
Seite sollen Sie in die Lage versetzt werden, so schnell und einfach wie
möglich so viel wie möglich aus dem System herauszuholen. Es wird eine
Herausforderung, die Sie schaffen werden. Gehen Sie gut gelaunt an R
heran, üben Sie Frustrationstoleranz, machen Sie Ihre Tastatut nicht
kaputt und sagen Sie sich immer wieder: Ich muss, ich will, ich kann.
:-)

Schauen Sie am Anfang mal Ihre Installation an. Wenn Sie R und R-Studio
das erste Mal installieren, dann können Sie das nach der Anleitung tu,
die Sie unter {[}Installation{]} finden. Dort erfahren Sie auch, wie Sie
\enquote{tinytex} mit R-Studio installieren und damit die Möglichkeit
haben, sehr schöne Berichte direkt in R-Studio setzen zu lassen
(\href{./files/Forschungsbericht_UE_3_AG_B.pdf}{zum Beispiel}). Damit
Sie gleich noch Literaturverweise und Literaturverzeichnis automatisiert
mitproduzieren können, finden Sie in der Installationsanleitung auch
gleich noch Zotero.

\hypertarget{programmieren-heisst-kopieren}{%
\section*{Programmieren heisst
kopieren}\label{programmieren-heisst-kopieren}}

\markright{Programmieren heisst kopieren}

Sie lernen mit der Zeit wie Sie Aufgaben in R lösen, indem Sie sich
Codeschnipsel suchen, diese kopieren und an Ihre Daten und Variablen
anpassen. Mit R zu arbeiten verlangt von Ihnen keine aktiven
Programmierkenntnisse. Sie müssen nur den richtigen Code (z.B. auf
\enquote{Stack Overflow}) finden und für sich anpassen. So arbeiten
übrigens die meisten Programmierer:innen in ihrem Alltag.

\bookmarksetup{startatroot}

\hypertarget{basics}{%
\chapter{Basics}\label{basics}}

\hypertarget{datenvisualisierung}{%
\section{Datenvisualisierung}\label{datenvisualisierung}}

\hypertarget{grundlagen-programmierung}{%
\section{Grundlagen Programmierung}\label{grundlagen-programmierung}}

\begin{itemize}
\tightlist
\item
  Funktionen
\item
  Objekte
\item
  Datentypen
\item
  Variablen, Datensätze
\item
  Pakete
\end{itemize}

\hypertarget{funktionen}{%
\subsection{Funktionen}\label{funktionen}}

\bookmarksetup{startatroot}

\hypertarget{Installation}{%
\chapter{Installation}\label{Installation}}

R ist ein sehr leistungsstarkes System für statistische Analysen. Sie
können R alleine herunterladen und den integrierten Editor nutzen, um
mit R zu arbeiten. Viel komfortabler ist es aber, die R-Umgebung
R-Studio zu nutzen, das Ihnen hilft, Ihre Daten und Variablen leichter
zu finden und in den R-Code zu integrieren, Sie finden Ihre Daten
leichter und die Dateien, mit denen Sie arbeiten. Zusätzlich hilft Ihnen
R-Studio dabei, Ihre Arbeit mit R verständlich zu dokumentieren. Dafür
nutzen Sie sehr komfortabel Markdown, eine ganz einfache
Kennzeichnungssprache, um Texte zu schreiben und zu formatieren. In
R-Studio haben Sie dafür sogar einen Formatierungseditor, der Ihnen auch
dafür Formatierungsschaltflächen zur Verfügung stellt, wie Sie sie von
Word und ähnlichen Programmen kennen. Sie können mit dem System sogar so
weit gehen, dass Sie Ihre Berichte komplett in R-Studio schreiben und am
Ende als schönen PDF-Bericht rauslassen können und sogar einfach als
Internetseite rauslassen können. Dabei werden Ihre Auswertungen in R als
Tabellen und Grafiken direkt integriert und sie müssen es nicht in ein
Word kopieren; UND, wenn Sie etwas an einer Variablen ändern oder etwas
am Datensatz korrigieren, würde das automatisch für den ganzen Bericht
geändert; UND mit einer für Sie erstellten Vorlage, haben Sie eine
schöne Formatierungsvorlage und brauchen sich auch um solche Dinge nicht
mehr gross zu kümmern. Dafür können/müssten Sie noch tinytex
installieren, müssen sich dann später aber nicht mehr drum kümmern. Eine
besondere Herausforderung wissenschaftlicher Arbeiten ist immer wieder
die Erstellung von korrekten Literaturnachweisen und eines
Quellenverzeichnisses. Auch da arbeitet das System für Sie. Für diese
Aufgabe empfehle ich Ihnen Citavi zu nutzen oder Zotero. Ich zeige Ihnen
daher noch, wie Sie Zotero installieren und nutzen können.

Hier können Sie sich die Installation des gesamten Systems im Video
anschauen.

\begin{Shaded}
\begin{Highlighting}[]
\NormalTok{vembedr}\SpecialCharTok{::}\FunctionTok{embed\_youtube}\NormalTok{(}\StringTok{"BVBvx99JMU0"}\NormalTok{) }\SpecialCharTok{|\textgreater{}}
\NormalTok{  vembedr}\SpecialCharTok{::}\FunctionTok{use\_rounded}\NormalTok{()}
\end{Highlighting}
\end{Shaded}

\hypertarget{downloadquellen}{%
\section{Downloadquellen}\label{downloadquellen}}

\hypertarget{r}{%
\subsection{R}\label{r}}

Laden Sie die neuste Version von R herunter:

\begin{itemize}
\item
  für Windows R base und am besten gleich noch R-tools R tools.
\item
  für Macs hier R-{[}\ldots{]}.pkg für ältere Macs mit Intel-Chips oder
  R-{[}\ldots{]}arm64.pkg für Macs mit M1 oder M2 (bei den Macs braucht
  es keine R-tools) .
\item
  für Linux R für Ihre Linuxversion .
\end{itemize}

\hypertarget{r-studio}{%
\subsection{R Studio}\label{r-studio}}

R-Studio finden Sie hier. Ihnen wird da ganz oben die Version angeboten,
die für Ihr System die Richtige ist!

\hypertarget{zotero}{%
\subsection{Zotero}\label{zotero}}

Mit Zotero können Sie Literatur verwalten, Verweise leicht automatisch
finden und einfach in RMarkdown-Dokumente integrieren oder auch
Verzeichnisse für Word erstellen. Zotero kann kostenlos genutzt werden.
Daher gehe ich hier eher darauf ein. Citavi ist auch gut und inzwischen
zum Glück auch für den Mac verfügbar. Der Nachteil ist nur, dass Sie
Citavi nicht mehr kostenlos nutzen, können, wenn Sie die UZH mal
verlassen und an eine andere Uni gehen, wo das Programm vielleicht nicht
kostenlos zur Verfügung gestellt wird. Daher gehe ich hier eher auf
Zotero ein.

Hier finden Sie Zotero. Installieren Sie dann gleich noch die
Erweiterung Dort finden Sie auch gleich die Anleitung für die
Installation der Erweiterung.

\hypertarget{reihenfolge-der-installation}{%
\section{Reihenfolge der
Installation}\label{reihenfolge-der-installation}}

\begin{enumerate}
\def\labelenumi{\arabic{enumi}.}
\tightlist
\item
  Installieren Sie erst R base
\item
  dann Rtools (nur auf Windows)
\item
  dann R-Studio
\item
  dann tinytex
\item
  dann Zotero
\item
  dann Zotero-better-bibtex
\item
  dann legen Sie den Ordner mit den ganzen Vorlagen bei sich ab.
\end{enumerate}

\hypertarget{installationsanleitung}{%
\section{Installationsanleitung}\label{installationsanleitung}}

R und R-Studio sind wie ganz normale Programe. Wie Sie die installieren,
zeige ich im Video, aber eigentlich brauchen Sie das nicht unbedingt.
Ein bischen anders ist die Installation von tinytex in R. Und dann zeige
ich Ihnen noch, wie Sie Pakete in R-Studio installieren.

\hypertarget{installation-von-tinytex}{%
\subsection{Installation von tinytex}\label{installation-von-tinytex}}

Gehen Sie in R-Studio oben im Menü auf den Menüpunkt \enquote{Tools} und
dort ganz oben auf \enquote{Install Packages \ldots{}}. Dann geben Sie
im leeren Eingabefeld \enquote{Packages} ein: \enquote{tinytex}. Wenn
Sie es richtig geschrieben haben, steht es auch gleich fett zur Auswahl.
Starten Sie die Installaion, indem Sie auf den Schalter
\enquote{Install} klicken.

Jetzt ist das Paket tinytex installiert, dass Ihnen die Arbeit abnimmt,
ein TeX-System zu installieren. Das machen Sie automatisch indem Sie in
der R-Studio-Console eingeben: \enquote{tinytex::install\_tinytex}. Dann
machen Sie etwas Schönes. Sie haben selbst bei schnelleren Rechnern Zeit
um sich einen Kaffee zu machen. Sie können aber auch in 20 Minuten etwas
stöbern, um die 20 Minuten Installation vorbeiziehen zu lassen. Lassen
Sie derweil Ihr System in Ruhe die Arbeit machen.

ODER, einfach diesen Code hier kopieren, in die R-(Studio)-Konsole
einfügen und mit Enter abschicken:

\begin{Shaded}
\begin{Highlighting}[]
\CommentTok{\# installiere das R{-}Paket tinytex}
\FunctionTok{install.packages}\NormalTok{(}\StringTok{"tinytex"}\NormalTok{) }

\CommentTok{\# installiere ein winziges LaTeX{-}System}
\NormalTok{tinytex}\SpecialCharTok{::}\FunctionTok{install\_tinytex}\NormalTok{() }
\end{Highlighting}
\end{Shaded}

\hypertarget{installation-von-paketen}{%
\subsection{Installation von Paketen}\label{installation-von-paketen}}

Die meisten Pakete können direkt von CRAN installiert werden. Das geht
in RStudio recht einfach über das Menü \enquote{Tools} -\textgreater{}
\enquote{Install Packages} und dann muss nur bei \enquote{Packages}
eingegeben werden, welche Pakete installiert werden sollen. Manchmal
gibt es die neusten Versionen eines Paketes nicht im Repository CRAN.
Dann müssen sie direkt von den Entwicklern installiert werden. Die
meisten Pakete werden dann über das sogenannte GitHub istalliert. Das
geht am einfachsten über Befehle nach dem Muster:
remotes::install\_github(\enquote{Entwicklername/Paket}). Häufig wird
dann von R noch gefragt, ob die Pakete und alle abhängigen Pakete
installiert werden sollen. Oder R fragt \enquote{Möchten Sie versuchen,
das Paket, welches eine Kompilierung erfordert, aus den Quelltexten zu
installieren?}. In der Regel ist es gut und richtig, wenn man das
bestätigt (Sie geben unten in der \enquote{Console} ein: \enquote{Yes}
und lassen R dann mal machen.). Manchmal gibt es aber einen Fehler weil
ein Paket nicht kompiliert werden kann oder so (häufiger auf Macs der
Fall). Dann ist es sinnvoll, dass man mal \enquote{n} für No eingibt.
Dann wählt R die letzte verfügbare binäre, also vorkompilierte Version.
Zum Beispiel ist das oft beim Paket \enquote{systemfonts} der Fall. Da
also besser mal beim zweiten Versuch das Kompilieren ablehnen, also auf
\enquote{no}.

Auch dieses Anleitung veraltet in der Regel schon nach einem Tag, weil
die Entwickler für R ständig an ihren Paketen basteln. Wenn also eine
Installation nicht funktioniert, ist es sinnvoll nach dem Paketnamen zu
googlen, also \enquote{r Paketname} googlen. In der Regel bekommen Sie
dann einen CRAN-Eintrag als einen der ersten Treffer. Wenn Sie da
draufgehen, sehen sie recht technische Einträge. Dort suchen Sie dann
\enquote{Vignette}. In den Vignetten steht eigentlich immer als erstes,
wie diese Pakete installiert werden sollen. Das kopieren Sie sich nach R
und schicken die Zeile einmal ab. Achten Sie darauf, ob vielleicht mal
ein Paket umbenannt wurde und es also Nachfolgepakete gibt oder so. Dann
nehmen Sie natürlich am besten die neue eventuell umbenannte Version.

\begin{Shaded}
\begin{Highlighting}[]
\CommentTok{\# installiere tidyverse:}
\FunctionTok{install.packages}\NormalTok{(}\StringTok{"tidyverse"}\NormalTok{) }

\DocumentationTok{\#\# installiere die neuste Version von tidycomm}

\CommentTok{\# Hier werden Sie gefragt, ob Sie versuchen möchten zu kompilieren und sagen "Yes" oder "no", wenn ersteres nicht klappt}

\CommentTok{\# jetzt installiere tidycomm}
\NormalTok{remotes}\SpecialCharTok{::}\FunctionTok{install\_github}\NormalTok{(}\StringTok{"joon{-}e/tidycomm"}\NormalTok{)}
\end{Highlighting}
\end{Shaded}

\hypertarget{die-vorlagen}{%
\section{Die Vorlagen}\label{die-vorlagen}}

Laden Sie diesen Zip-Ornder runter und speichern Sie ihn irgendwo (am
besten in einem SWITCH-Drive-Ordner). Entpacken Sie den Ordner (meistens
reicht der Doppelklick). In dem Ordner finden sie einige Dateien. Eine
Datei heisst \enquote{Bericht.Rproj}. Öffnen Sie die mal. Dann finden
Sie im Panel rechts unten einige Dateien. Öffnen Sie
\enquote{index.Rmd}. Da ist das Setup drin und die Einleitung. Wenn Sie
die öffnen und noch nicht alle Pakete installiert haben, die dafür
benötigt werden, dann zeigt Ihnen R-Studio oben im Hauptfenster ein
schmales gelbes Band an, in dem Sie darauf hingewiesen werden, dass
nicht alle Pakete installiert sind. Dort können Sie die Installation
direkt und automatisch starten lassen. Das dauert dann eine Weile, aber
irgendwann ist es fertig. :-) Wenn die schrittweise Intallation
glattgelaufen ist, sollten Sie rechts oben einen Reiter sehen, der
\enquote{Build} heisst. Wenn Sie darauf gehen, sehen Sie einen neuen
Schalter mit einem Hammer darauf. Klicken Sie da den Pfeil nach unten
und dann auf \enquote{bookdown::pdf\_book} (Wenn hier nur Website steht,
dann starten Sie R nach der Installation nochmal neu). Dann warten Sie
(beim ersten Mal mit viel Geduld über 15-30 Minuten!). Am Ende sollte
Ihnen ein PDF angezeight werden.

\bookmarksetup{startatroot}

\hypertarget{fehler-erkennen-und-luxf6sen}{%
\chapter{Fehler erkennen und lösen}\label{fehler-erkennen-und-luxf6sen}}

\hypertarget{vorgehen-bei-fehlern}{%
\section{Vorgehen bei Fehlern}\label{vorgehen-bei-fehlern}}

\emph{Problemlösungsworkflow (je Schritt 15 Minuten)}

\begin{enumerate}
\def\labelenumi{\arabic{enumi}.}
\item
  Fragen Sie erst sich, was wohl die Fehlermeldung heissen könnte und
  was anders ist zu Vorlagen und Beispielen (und google bzw. auf Stack
  Overflow suchen),
\item
  dann Ihre AG-Kollegen,
\item
  bei Task-Force-Aufgaben Ihre Task-Force-Kolleg:innen (zB TF D auf
  MS-Teams)
\item
  dann Ihre AG E,
\item
  dann im Forum auf OLAT,
\item
  dann direkt an
  \href{mailto:b.fretwurst@ikmz.uzh.ch}{\nolinkurl{b.fretwurst@ikmz.uzh.ch}}
  oder auf Teams \enquote{Fretwurst} oder \enquote{Nadia},
\item
  Stack Overflow mit \enquote{reprex} bzw. \enquote{mwe},
\item
  Externe direkt (keine Mails an Martin Wettstein). Eher fragt Fretwurst
  bei Paketautoren.
\end{enumerate}

\hypertarget{bei-fragen-zu-fehlern-mitzugeben}{%
\section{Bei Fragen zu Fehlern
mitzugeben}\label{bei-fragen-zu-fehlern-mitzugeben}}

Wenn Sie Fehlermeldungen bekommen, senden Sie nicht einfach nur
Screenshots. Damit kann niemand etwas anfangen. Wenn Ihnen geholfen
werden soll, müssen Sie die R-Zeilen (bzw. die R-Script-Datei) sowie den
zugrundeliegenden Datensatz mitliefern und mitteilen, wo der Fehler
auftritt.

Zu Ihren Fragen liefern Sie also bitte immer:

\begin{enumerate}
\def\labelenumi{\arabic{enumi}.}
\item
  die Datendatei (als .RDS, .RData oder Excel-Datei),
\item
  Betroffener Befehl muss mitgeliefert werden als Text. Im Idealfall die
  .Rmd mit Zeilenhinweis (lokalisieren Sie den Fehler innerhalb einer
  Pipe).
\item
  Den Fehleroutput als Text und zusätzlich gerne als Screenshot.
\item
  Was schon probiert wurde.
\end{enumerate}

\hypertarget{typische-fehler-erkennen-und-beheben}{%
\section{Typische Fehler erkennen und
beheben}\label{typische-fehler-erkennen-und-beheben}}

Die häufigsten Fehler und Lösungen sind:

\begin{enumerate}
\def\labelenumi{\arabic{enumi}.}
\item
  Pakete sind nicht installiert: Tools -\textgreater{} \enquote{Install
  Packages} Paket installieren
\item
  Pakete sind nicht geladen: Paket laden library(Paketname) und
  index.Rmd nutzen, um alle nötigen Pakete in der richtigen Reihenfolge
  zu laden
\item
  Datensätze können nicht gefunden werden: Den richtigen Pfad einstellen
  mit Unterordner zB DATEN\_BF \textless-
  readRDS(\enquote{Daten/DATEN\_BF.RDS})
\item
  Datenobjekt wird nicht gefunden, weil es noch nicht geladen wurde:
  Daten laden zB DATEN\_BF \textless-
  readRDS(\enquote{Daten/DATEN\_BF.RDS})
\item
  Datenobjekt wird nicht gefunden, weil es falsch geschrieben ist:
  Kontrollieren Sie auch Gross-Klein-Schreibung
\item
  Variable wird nicht gefunden, weil sie falsch geschrieben ist:
  Schreiben Sie den Variablennamen neu und schauen Sie ob sie ihn in der
  Hilfsliste sehen.
\item
  Kommata oder Klammern fehlen oder sind zu viel: Es wird links in R ein
  roter Kreis mit weissem x angezeigt. Wenn Sie da drauf gehen, sagt
  Ihnen R-Studio in der Regel was fehlt oder zu viel ist.
\item
  Es werden mehrere Zeilen grün ausgeführt, aber es passiert nichts
  weiter: Schauen Sie ob am Ende ein Magritter (\%\textgreater\%) zu
  viel ist.
\item
  Es kann \enquote{eine Funktion nicht gefunden} werden, weil ein Paket
  fehlt oder die Funktion ist falsch geschrieben ist: Googlen Sie die
  Funktion und schauen, aus welchem Paket die ist. Sie können dann
  probieren, ob¬ es mit Paketname::Funktion geht zB janitor::percent.
\item
  Variablen sind vom falschen Typ: Wandeln Sie den Typ um. ZB: DATEN\_BF
  \%\textgreater\% mutate(Nachricht = as.numeric(Nachricht)).
\end{enumerate}

Googlen Sie Ihre Fehlermeldungen oder Funktionen, die Fehler erzeugen
und schauen sich den Syntax an, wie die Funktionen geschrieben sein
sollten. Üben Sie sich in Frustrationstolleranz. Haben Sie geduld und
suchen Sie den Fehler im Detail.

\href{https://www.uni-muenster.de/imperia/md/content/psyifp/ae_holling/r-fehlermeldungen.pdf}{Ein
Kollege gibt hier ein paar Hinweise wie Fehler in R lesbar sind in einer
PDF-Datei.} Hier habe ich noch ein Video mit typischen Fehlern und
Lösungen:

\bookmarksetup{startatroot}

\hypertarget{relitest}{%
\chapter{Relitest}\label{relitest}}

Dazu gibt es auch ein Video:

\begin{verbatim}
## -- Attaching packages ------------------------------------------------- tidyverse 1.3.2 --
## v ggplot2 3.4.0      v purrr   1.0.1 
## v tibble  3.1.8      v dplyr   1.0.10
## v tidyr   1.2.1      v stringr 1.5.0 
## v readr   2.1.3      v forcats 0.5.2 
## -- Conflicts ---------------------------------------------------- tidyverse_conflicts() --
## x dplyr::filter() masks stats::filter()
## x dplyr::lag()    masks stats::lag()
\end{verbatim}

Hier finden Sie den im Video erwähnten gezippten Ordner mit der
Relitest.Rmd. Den Ordner müssen Sie entpacken und an eine gute Stelle in
ihrem System tun. In dem Ordner wo sich diese Datei
\enquote{Relitest.Rmd} befindet, muss es einen Unterordner mit der
Bezeichnung \enquote{Daten} geben, wo die Excel-Dateien
\enquote{RelidatenCoder1.xlsx} usw. liegen. s \#\# Relidaten
Zusammenbinden

Sie werden die Inhaltsanalyse in Excel kodieren. Die Exceldateien müssen
dann in R importiert werden. Das geht zB mit \textbf{read\_xlsx} des
Pakets \textbf{read.xl}. Wie Sie sehen werden die Exceldateien der Coder
(im Beispiel 1 bis 3) jeweils in R-Dateien eingelesen (RelitestCoder1
\textless- \ldots) und dann mit dem Befehl rbind (steht für Rows
aneinander binden) zu einer Datei zusammengefügt, die dann angeschaut
oder für Relitests weiterverarbeitet werden kann.

\begin{Shaded}
\begin{Highlighting}[]
\CommentTok{\# Lade alle Relitestdateien aus dem Ordner "Daten" (den es geben muss mit allen Relidateien):}

\NormalTok{RelitestCoder1 }\OtherTok{\textless{}{-}}\NormalTok{ readxl}\SpecialCharTok{::}\FunctionTok{read\_excel}\NormalTok{(}\StringTok{"Daten/RelidatenCoder1.xlsx"}\NormalTok{)}
\NormalTok{RelitestCoder2 }\OtherTok{\textless{}{-}}\NormalTok{ readxl}\SpecialCharTok{::}\FunctionTok{read\_excel}\NormalTok{(}\StringTok{"Daten/RelidatenCoder2.xlsx"}\NormalTok{)}
\NormalTok{RelitestCoder3 }\OtherTok{\textless{}{-}}\NormalTok{ readxl}\SpecialCharTok{::}\FunctionTok{read\_excel}\NormalTok{(}\StringTok{"Daten/Klausis erster Relitest.xlsx"}\NormalTok{)}

\CommentTok{\# Binde die Relidatensätze der Coder:innen zu einem Datenobjekt "RelitestGesamt zusammen:}
\NormalTok{Relitest\_Gesamt }\OtherTok{\textless{}{-}} \FunctionTok{rbind}\NormalTok{(RelitestCoder1, RelitestCoder2, RelitestCoder3) }

\NormalTok{Relitest\_Gesamt }\SpecialCharTok{|\textgreater{}} 
\NormalTok{  dplyr}\SpecialCharTok{::}\FunctionTok{select}\NormalTok{(CODER, CU) }\SpecialCharTok{|\textgreater{}} 
\NormalTok{  dplyr}\SpecialCharTok{::}\FunctionTok{mutate}\NormalTok{(}\AttributeTok{Fehler =} \FunctionTok{ifelse}\NormalTok{(}\FunctionTok{is.na}\NormalTok{(CODER) }\SpecialCharTok{|} \FunctionTok{is.na}\NormalTok{(CU), }\StringTok{"FEHLER! Muss in der RelitestGesamt.xlsx repariert und als Relitest\_bearbeitet.xlsx gespeichert werden oder der:die Coder:in muss es überarbeiten"}\NormalTok{, }\StringTok{"alles gut"}\NormalTok{)) }\SpecialCharTok{|\textgreater{}} 
\NormalTok{  dplyr}\SpecialCharTok{::}\FunctionTok{count}\NormalTok{(Fehler)}
\DocumentationTok{\#\# \# A tibble: 2 x 2}
\DocumentationTok{\#\#   Fehler                                                                                 n}
\DocumentationTok{\#\#   \textless{}chr\textgreater{}                                                                              \textless{}int\textgreater{}}
\DocumentationTok{\#\# 1 alles gut                                                                             15}
\DocumentationTok{\#\# 2 FEHLER! Muss in der RelitestGesamt.xlsx repariert und als Relitest\_bearbeitet.xls\textasciitilde{}     7}

\CommentTok{\# hier schreibe ich die Daten mal aus R raus in Excel, damit ich sie in Excel besser angucken kann und dort easy bereinige. Kann so gemacht werden oder alles in R, was schicker ist, aber nicht so praktisch.}

\NormalTok{writexl}\SpecialCharTok{::}\FunctionTok{write\_xlsx}\NormalTok{(Relitest\_Gesamt, }\StringTok{"Daten/Relitest\_Gesamt.xlsx"}\NormalTok{)}
\end{Highlighting}
\end{Shaded}

\hypertarget{relitest-1}{%
\section{Relitest}\label{relitest-1}}

Den Relitest selbst kann man mit test\_icr des Pakets tidycomm machen.
Dazu werden zunächst die Daten geladen (im Beispiel
\enquote{RelitestDaten}) und dann mit der Pipe (neu statt
\%\textgreater\% der Pipoperator \textbar\textgreater{} ) der Befehl
test\_icr ausgeführt. An erster Stelle steht die Variable für die Coding
Unit (CU), die im Datensatz RelitestBeispiele einfach \enquote{CU}
heisst. Als nächstes folgt die Variable für den Coder (im Beispiel
einfach \enquote{CODER}). In den Variablen CU und CODER dürfen nur
Zahlen stehen! Also muss jeder Kodierer eine Nummer eingeben und auch
die Artikel im Relimaterial nummeriert sein. Dann können Sie einzelne
Variablennamen anfügen, für die die Reliwerte ausgewertet werden sollen
oooooder Sie lassen das weg und es werden für alle Variablen Reliwerte
berechnet. Wenn Sie fehlende Werte im Datensatz haben, dann sollten Sie
am Ende \enquote{na.omit = TRUE} stehen haben, sonst meckert R, dass
Fehlende Werte gefunden wurden.

\begin{Shaded}
\begin{Highlighting}[]
\CommentTok{\# hier lade ich die Relidaten aus Excel wieder rein in R}
\NormalTok{RelitestDaten }\OtherTok{\textless{}{-}}\NormalTok{ readxl}\SpecialCharTok{::}\FunctionTok{read\_xlsx}\NormalTok{(}\StringTok{"Daten/Relitest\_bearbeitet.xlsx"}\NormalTok{)}

\CommentTok{\# Sie können mit dem folgenden Befehl Variablen umkodieren und gleich in eckigen Klammern labeln }
\NormalTok{RelitestDaten }\OtherTok{\textless{}{-}}\NormalTok{ RelitestDaten }\SpecialCharTok{|\textgreater{}} 
  \FunctionTok{mutate}\NormalTok{(}\AttributeTok{Worte\_gr =}\NormalTok{ sjmisc}\SpecialCharTok{::}\FunctionTok{rec}\NormalTok{(Worte, }\AttributeTok{rec =} 
                       \StringTok{"min:100 = 1 [Kurzmeldung]; }
\StringTok{                        101:150 = 2 [kurzer Artikel]; }
\StringTok{                        151:300 = 3 [länger]; }
\StringTok{                        301:500 = 4 [lang]; }
\StringTok{                        501:max = 5 [Langtext]"}\NormalTok{)) }\SpecialCharTok{|\textgreater{}} 
    \FunctionTok{relocate}\NormalTok{(Worte\_gr, }\AttributeTok{.after =}\NormalTok{ Worte) }\CommentTok{\# hier sortiere ich die neue Variable direkt hinter die alte}

\NormalTok{RelitestDaten }\SpecialCharTok{|\textgreater{}} 
\NormalTok{  sjmisc}\SpecialCharTok{::}\FunctionTok{frq}\NormalTok{(Worte\_gr) }\CommentTok{\# machen wir mal ne Häufigkeitsauszählung für "Worte\_gr"}
\DocumentationTok{\#\# Worte\_gr \textless{}numeric\textgreater{} }
\DocumentationTok{\#\# \# total N=190 valid N=190 mean=3.61 sd=1.20}
\DocumentationTok{\#\# }
\DocumentationTok{\#\# Value |          Label |  N | Raw \% | Valid \% | Cum. \%}
\DocumentationTok{\#\# {-}{-}{-}{-}{-}{-}{-}{-}{-}{-}{-}{-}{-}{-}{-}{-}{-}{-}{-}{-}{-}{-}{-}{-}{-}{-}{-}{-}{-}{-}{-}{-}{-}{-}{-}{-}{-}{-}{-}{-}{-}{-}{-}{-}{-}{-}{-}{-}{-}{-}{-}{-}{-}{-}}
\DocumentationTok{\#\#     1 |    Kurzmeldung | 19 | 10.00 |   10.00 |  10.00}
\DocumentationTok{\#\#     2 | kurzer Artikel | 19 | 10.00 |   10.00 |  20.00}
\DocumentationTok{\#\#     3 |         länger | 18 |  9.47 |    9.47 |  29.47}
\DocumentationTok{\#\#     4 |           lang | 96 | 50.53 |   50.53 |  80.00}
\DocumentationTok{\#\#     5 |       Langtext | 38 | 20.00 |   20.00 | 100.00}
\DocumentationTok{\#\#  \textless{}NA\textgreater{} |           \textless{}NA\textgreater{} |  0 |  0.00 |    \textless{}NA\textgreater{} |   \textless{}NA\textgreater{}}

\NormalTok{RelitestDaten }\OtherTok{\textless{}{-}}\NormalTok{ RelitestDaten }\SpecialCharTok{|\textgreater{}} 
\NormalTok{  dplyr}\SpecialCharTok{::}\FunctionTok{mutate}\NormalTok{(}\AttributeTok{Voyer\_r =}\NormalTok{ sjmisc}\SpecialCharTok{::}\FunctionTok{rec}\NormalTok{(Voyeur, }\AttributeTok{rec =} \StringTok{"0 = 0 [kommt nicht vor]; 1:3 = 1 [kommt vor]; else = copy"}\NormalTok{)) }\SpecialCharTok{|\textgreater{}} 
\NormalTok{  dplyr}\SpecialCharTok{::}\FunctionTok{relocate}\NormalTok{(Voyer\_r, }\AttributeTok{.after =}\NormalTok{ Voyeur)}

\NormalTok{RelitestDaten }\SpecialCharTok{|\textgreater{}} 
\NormalTok{  sjmisc}\SpecialCharTok{::}\FunctionTok{frq}\NormalTok{(Voyer\_r)}
\DocumentationTok{\#\# Voyer\_r \textless{}numeric\textgreater{} }
\DocumentationTok{\#\# \# total N=190 valid N=190 mean=0.44 sd=0.50}
\DocumentationTok{\#\# }
\DocumentationTok{\#\# Value |           Label |   N | Raw \% | Valid \% | Cum. \%}
\DocumentationTok{\#\# {-}{-}{-}{-}{-}{-}{-}{-}{-}{-}{-}{-}{-}{-}{-}{-}{-}{-}{-}{-}{-}{-}{-}{-}{-}{-}{-}{-}{-}{-}{-}{-}{-}{-}{-}{-}{-}{-}{-}{-}{-}{-}{-}{-}{-}{-}{-}{-}{-}{-}{-}{-}{-}{-}{-}{-}}
\DocumentationTok{\#\#     0 | kommt nicht vor | 106 | 55.79 |   55.79 |  55.79}
\DocumentationTok{\#\#     1 |       kommt vor |  84 | 44.21 |   44.21 | 100.00}
\DocumentationTok{\#\#  \textless{}NA\textgreater{} |            \textless{}NA\textgreater{} |   0 |  0.00 |    \textless{}NA\textgreater{} |   \textless{}NA\textgreater{}}

\CommentTok{\# hier kommt dann der eigentliche Relitest. in test\_icr muss ganz vorne der Variblenname der Kodiereinheit (CU) stehen und danach für die CODER. }

\NormalTok{RelitestOutput }\OtherTok{\textless{}{-}}\NormalTok{ RelitestDaten }\SpecialCharTok{|\textgreater{}} 
\NormalTok{  tidycomm}\SpecialCharTok{::}\FunctionTok{test\_icr}\NormalTok{(}\AttributeTok{unit\_var =}\NormalTok{ CU, }\AttributeTok{coder\_var =}\NormalTok{ CODER, Medium}\SpecialCharTok{:}\NormalTok{Kurios, }\AttributeTok{kripp\_alpha =} \ConstantTok{TRUE}\NormalTok{, }\AttributeTok{lotus =} \ConstantTok{TRUE}\NormalTok{, }\AttributeTok{s\_lotus =} \ConstantTok{TRUE}\NormalTok{, }\AttributeTok{na.omit =} \ConstantTok{TRUE}\NormalTok{)}

\NormalTok{RelitestOutput }\CommentTok{\# hier den Output des Relitest mal schnell in R angucken}
\DocumentationTok{\#\# \# A tibble: 39 x 10}
\DocumentationTok{\#\#   Variable   n\_Units n\_Coders n\_Categories Level   Agreement Holst\textasciitilde{}1 Kripp\textasciitilde{}2 Lotus S\_Lotus}
\DocumentationTok{\#\#   \textless{}chr\textgreater{}        \textless{}int\textgreater{}    \textless{}int\textgreater{}        \textless{}int\textgreater{} \textless{}chr\textgreater{}       \textless{}dbl\textgreater{}   \textless{}dbl\textgreater{}   \textless{}dbl\textgreater{} \textless{}dbl\textgreater{}   \textless{}dbl\textgreater{}}
\DocumentationTok{\#\# 1 Medium          10       19            5 nominal       1     1       1     1       1    }
\DocumentationTok{\#\# 2 ThemSchwer      10       19            6 nominal       0.5   0.860   0.794 0.911   0.893}
\DocumentationTok{\#\# 3 Titel           10       19           36 nominal       0.1   0.665   0.643 0.795   0.789}
\DocumentationTok{\#\# 4 Worte           10       19           42 nominal       0     0.378   0.343 0.589   0.579}
\DocumentationTok{\#\# 5 Worte\_gr        10       19            5 nominal       0.2   0.850   0.780 0.921   0.901}
\DocumentationTok{\#\# 6 ArtLang         10       19            2 nominal       0.7   0.959   0.915 0.979   0.958}
\DocumentationTok{\#\# \# ... with 33 more rows, and abbreviated variable names 1: Holstis\_CR,}
\DocumentationTok{\#\# \#   2: Krippendorffs\_Alpha}

\CommentTok{\# Das ist im Prinzip das Gleiche wie oben, aber für Worte wird das Skalenniveau "interval" angegeben und für Artikellänge "ordinal"}
\NormalTok{RelitestOutput }\OtherTok{\textless{}{-}}\NormalTok{ RelitestDaten }\SpecialCharTok{|\textgreater{}} 
\NormalTok{tidycomm}\SpecialCharTok{::}\FunctionTok{test\_icr}\NormalTok{(}\AttributeTok{unit\_var =}\NormalTok{ CU, }\AttributeTok{coder\_var =}\NormalTok{ CODER, Medium}\SpecialCharTok{:}\NormalTok{Kurios, }\CommentTok{\# }
           \AttributeTok{levels =} \FunctionTok{c}\NormalTok{(}\AttributeTok{Worte =} \StringTok{"interval"}\NormalTok{, }\AttributeTok{ArtLang =} \StringTok{"ordinal"}\NormalTok{), }\CommentTok{\# hier die Skalnenniveaus; nominal ist Standard}
           \AttributeTok{kripp\_alpha =} \ConstantTok{TRUE}\NormalTok{, }\AttributeTok{lotus =} \ConstantTok{TRUE}\NormalTok{, }\AttributeTok{s\_lotus =} \ConstantTok{TRUE}\NormalTok{, }\AttributeTok{agreement =} \ConstantTok{FALSE}\NormalTok{, }\AttributeTok{holsti =} \ConstantTok{FALSE}\NormalTok{, }\AttributeTok{na.omit =} \ConstantTok{TRUE}\NormalTok{) }\SpecialCharTok{|\textgreater{}} 
\NormalTok{    dplyr}\SpecialCharTok{::}\FunctionTok{mutate\_if}\NormalTok{(is.numeric, round, }\AttributeTok{digits =} \DecValTok{2}\NormalTok{) }\SpecialCharTok{|\textgreater{}}  \CommentTok{\# hiermit noch auf zwei Stellen runden {-} reicht jas}
\NormalTok{    dplyr}\SpecialCharTok{::}\FunctionTok{mutate}\NormalTok{(}\AttributeTok{S\_Lotus =} \FunctionTok{ifelse}\NormalTok{(}\FunctionTok{is.na}\NormalTok{(S\_Lotus), }\DecValTok{1}\NormalTok{, S\_Lotus)) }\CommentTok{\# Wenn alle Coder:innen nur eine und dieselbe Kategorie gewählt haben, ist S\_Lotus nicht definiert, aber 1}


\NormalTok{RelitestOutput }\SpecialCharTok{|\textgreater{}} 
\NormalTok{  kableExtra}\SpecialCharTok{::}\FunctionTok{kbl}\NormalTok{() }\SpecialCharTok{|\textgreater{}} \CommentTok{\# und hier noch hübsch und dann bunt mit kbl}
\NormalTok{  kableExtra}\SpecialCharTok{::}\FunctionTok{kable\_material}\NormalTok{() }\SpecialCharTok{|\textgreater{}} 
\NormalTok{    kableExtra}\SpecialCharTok{::}\FunctionTok{column\_spec}\NormalTok{(}\DecValTok{6}\NormalTok{, }\AttributeTok{color =} \FunctionTok{ifelse}\NormalTok{(RelitestOutput}\SpecialCharTok{$}\NormalTok{Krippendorffs\_Alpha }\SpecialCharTok{\textgreater{}} \FloatTok{0.3} \SpecialCharTok{\&}\NormalTok{ RelitestOutput}\SpecialCharTok{$}\NormalTok{Krippendorffs\_Alpha }\SpecialCharTok{\textless{}} \FloatTok{0.75}\NormalTok{, }\StringTok{"black"}\NormalTok{, }\StringTok{"white"}\NormalTok{),}
              \AttributeTok{background =}\NormalTok{ kableExtra}\SpecialCharTok{::}\FunctionTok{spec\_color}\NormalTok{(RelitestOutput}\SpecialCharTok{$}\NormalTok{Krippendorffs\_Alpha[}\DecValTok{0}\SpecialCharTok{:}\DecValTok{100}\NormalTok{], }\AttributeTok{begin =} \DecValTok{0}\NormalTok{, }\AttributeTok{end =} \DecValTok{1}\NormalTok{, }\AttributeTok{option =} \StringTok{"turbo"}\NormalTok{, }\AttributeTok{direction =} \SpecialCharTok{{-}}\DecValTok{1}\NormalTok{)) }\SpecialCharTok{|\textgreater{}} 
\NormalTok{    kableExtra}\SpecialCharTok{::}\FunctionTok{column\_spec}\NormalTok{(}\DecValTok{7}\NormalTok{, }\AttributeTok{color =} \FunctionTok{ifelse}\NormalTok{(RelitestOutput}\SpecialCharTok{$}\NormalTok{Lotus }\SpecialCharTok{\textgreater{}} \FloatTok{0.6} \SpecialCharTok{\&}\NormalTok{ RelitestOutput}\SpecialCharTok{$}\NormalTok{Lotus }\SpecialCharTok{\textless{}} \FloatTok{0.9}\NormalTok{, }\StringTok{"black"}\NormalTok{, }\StringTok{"white"}\NormalTok{),}
              \AttributeTok{background =}\NormalTok{ kableExtra}\SpecialCharTok{::}\FunctionTok{spec\_color}\NormalTok{(RelitestOutput}\SpecialCharTok{$}\NormalTok{Lotus[}\DecValTok{0}\SpecialCharTok{:}\DecValTok{100}\NormalTok{], }\AttributeTok{begin =} \DecValTok{0}\NormalTok{, }\AttributeTok{end =} \DecValTok{1}\NormalTok{, }\AttributeTok{option =} \StringTok{"turbo"}\NormalTok{, }\AttributeTok{direction =} \SpecialCharTok{{-}}\DecValTok{1}\NormalTok{)) }\SpecialCharTok{|\textgreater{}} 
\NormalTok{    kableExtra}\SpecialCharTok{::}\FunctionTok{column\_spec}\NormalTok{(}\DecValTok{8}\NormalTok{, }\AttributeTok{color =} \FunctionTok{ifelse}\NormalTok{(RelitestOutput}\SpecialCharTok{$}\NormalTok{S\_Lotus }\SpecialCharTok{\textgreater{}} \FloatTok{0.6} \SpecialCharTok{\&}\NormalTok{ RelitestOutput}\SpecialCharTok{$}\NormalTok{S\_Lotus }\SpecialCharTok{\textless{}} \FloatTok{0.9}\NormalTok{, }\StringTok{"black"}\NormalTok{, }\StringTok{"white"}\NormalTok{),}
              \AttributeTok{background =}\NormalTok{ kableExtra}\SpecialCharTok{::}\FunctionTok{spec\_color}\NormalTok{(RelitestOutput}\SpecialCharTok{$}\NormalTok{S\_Lotus[}\DecValTok{0}\SpecialCharTok{:}\DecValTok{100}\NormalTok{], }\AttributeTok{begin =} \DecValTok{0}\NormalTok{, }\AttributeTok{end =} \DecValTok{1}\NormalTok{, }\AttributeTok{option =} \StringTok{"turbo"}\NormalTok{, }\AttributeTok{direction =} \SpecialCharTok{{-}}\DecValTok{1}\NormalTok{))}
\DocumentationTok{\#\# Warning in ensure\_len\_latex(background, nrows, off, include\_thead, "white", : The number}
\DocumentationTok{\#\# of provided values in background does not equal to the number of rows.}

\DocumentationTok{\#\# Warning in ensure\_len\_latex(background, nrows, off, include\_thead, "white", : The number}
\DocumentationTok{\#\# of provided values in background does not equal to the number of rows.}

\DocumentationTok{\#\# Warning in ensure\_len\_latex(background, nrows, off, include\_thead, "white", : The number}
\DocumentationTok{\#\# of provided values in background does not equal to the number of rows.}
\end{Highlighting}
\end{Shaded}

\begin{table}
\centering
\begin{tabular}[t]{l|r|r|r|l|>{}r|>{}r|>{}r}
\hline
Variable & n\_Units & n\_Coders & n\_Categories & Level & Krippendorffs\_Alpha & Lotus & S\_Lotus\\
\hline
Medium & 10 & 19 & 5 & nominal & \cellcolor[HTML]{30123B}{\textcolor{white}{1.00}} & \cellcolor[HTML]{30123B}{\textcolor{white}{1.00}} & \cellcolor[HTML]{30123B}{\textcolor{white}{1.00}}\\
\hline
ThemSchwer & 10 & 19 & 6 & nominal & \cellcolor[HTML]{3AA3FC}{\textcolor{white}{0.79}} & \cellcolor[HTML]{3D9EFE}{\textcolor{white}{0.91}} & \cellcolor[HTML]{4294FF}{\textcolor{black}{0.89}}\\
\hline
Titel & 10 & 19 & 36 & nominal & \cellcolor[HTML]{2AEFA1}{\textcolor{black}{0.64}} & \cellcolor[HTML]{92FF47}{\textcolor{black}{0.79}} & \cellcolor[HTML]{27EEA4}{\textcolor{black}{0.79}}\\
\hline
Worte & 10 & 19 & 42 & interval & \cellcolor[HTML]{1CCDD8}{\textcolor{black}{0.72}} & \cellcolor[HTML]{AF1801}{\textcolor{white}{0.59}} & \cellcolor[HTML]{FE932A}{\textcolor{white}{0.58}}\\
\hline
Worte\_gr & 10 & 19 & 5 & nominal & \cellcolor[HTML]{35ABF8}{\textcolor{white}{0.78}} & \cellcolor[HTML]{448FFE}{\textcolor{white}{0.92}} & \cellcolor[HTML]{458AFC}{\textcolor{white}{0.90}}\\
\hline
ArtLang & 10 & 19 & 2 & ordinal & \cellcolor[HTML]{434EBA}{\textcolor{white}{0.92}} & \cellcolor[HTML]{3D358B}{\textcolor{white}{0.98}} & \cellcolor[HTML]{4146AC}{\textcolor{white}{0.96}}\\
\hline
Datum & 10 & 19 & 4 & nominal & \cellcolor[HTML]{1AE4B6}{\textcolor{black}{0.67}} & \cellcolor[HTML]{448FFE}{\textcolor{white}{0.92}} & \cellcolor[HTML]{4294FF}{\textcolor{black}{0.89}}\\
\hline
Format & 10 & 19 & 4 & nominal & \cellcolor[HTML]{467DF4}{\textcolor{white}{0.85}} & \cellcolor[HTML]{4454C3}{\textcolor{white}{0.96}} & \cellcolor[HTML]{4451BF}{\textcolor{white}{0.95}}\\
\hline
Bilder & 10 & 19 & 13 & nominal & \cellcolor[HTML]{28BCEB}{\textcolor{white}{0.75}} & \cellcolor[HTML]{1DE7B2}{\textcolor{black}{0.85}} & \cellcolor[HTML]{1FC9DD}{\textcolor{black}{0.84}}\\
\hline
Nutzen\_ig & 10 & 19 & 4 & nominal & \cellcolor[HTML]{F1CB3A}{\textcolor{black}{0.37}} & \cellcolor[HTML]{3CF58E}{\textcolor{black}{0.83}} & \cellcolor[HTML]{35F394}{\textcolor{black}{0.78}}\\
\hline
Alltagswelt & 10 & 19 & 3 & nominal & \cellcolor[HTML]{9B0F01}{\textcolor{white}{0.05}} & \cellcolor[HTML]{92FF47}{\textcolor{black}{0.79}} & \cellcolor[HTML]{B9F635}{\textcolor{black}{0.69}}\\
\hline
Nah\_Welt & 10 & 19 & 2 & nominal & \cellcolor[HTML]{E5D938}{\textcolor{black}{0.40}} & \cellcolor[HTML]{1FC9DD}{\textcolor{black}{0.88}} & \cellcolor[HTML]{43F787}{\textcolor{black}{0.77}}\\
\hline
Nah\_EU & 10 & 19 & 2 & nominal & \cellcolor[HTML]{22EBAA}{\textcolor{black}{0.65}} & \cellcolor[HTML]{4682F8}{\textcolor{white}{0.93}} & \cellcolor[HTML]{2CB7F0}{\textcolor{black}{0.86}}\\
\hline
Nah\_CH & 10 & 19 & 2 & nominal & \cellcolor[HTML]{D9E436}{\textcolor{black}{0.42}} & \cellcolor[HTML]{2AEFA1}{\textcolor{black}{0.84}} & \cellcolor[HTML]{C6F034}{\textcolor{black}{0.68}}\\
\hline
Nah\_Kant & 10 & 19 & 2 & nominal & \cellcolor[HTML]{EECF3A}{\textcolor{black}{0.38}} & \cellcolor[HTML]{1FC9DD}{\textcolor{black}{0.88}} & \cellcolor[HTML]{55FA76}{\textcolor{black}{0.76}}\\
\hline
Nah\_Stadt & 10 & 19 & 2 & nominal & \cellcolor[HTML]{CDEC34}{\textcolor{black}{0.44}} & \cellcolor[HTML]{18DEC0}{\textcolor{black}{0.86}} & \cellcolor[HTML]{88FF4E}{\textcolor{black}{0.73}}\\
\hline
Nah\_Gem & 10 & 19 & 2 & nominal & \cellcolor[HTML]{FDAE35}{\textcolor{black}{0.32}} & \cellcolor[HTML]{4682F8}{\textcolor{white}{0.93}} & \cellcolor[HTML]{25C0E7}{\textcolor{black}{0.85}}\\
\hline
Nah\_nix & 10 & 19 & 2 & nominal & \cellcolor[HTML]{A91601}{\textcolor{white}{0.07}} & \cellcolor[HTML]{4143A7}{\textcolor{white}{0.97}} & \cellcolor[HTML]{4451BF}{\textcolor{white}{0.95}}\\
\hline
Naehe & 10 & 19 & 16 & nominal & \cellcolor[HTML]{E5D938}{\textcolor{black}{0.40}} & \cellcolor[HTML]{F76F1A}{\textcolor{black}{0.66}} & \cellcolor[HTML]{EBD339}{\textcolor{black}{0.64}}\\
\hline
Voyeur & 10 & 19 & 4 & nominal & \cellcolor[HTML]{FEA130}{\textcolor{white}{0.30}} & \cellcolor[HTML]{FEA130}{\textcolor{black}{0.69}} & \cellcolor[HTML]{FEA130}{\textcolor{white}{0.59}}\\
\hline
Voyer\_r & 10 & 19 & 2 & nominal & \cellcolor[HTML]{FEA732}{\textcolor{black}{0.31}} & \cellcolor[HTML]{A4FC3C}{\textcolor{black}{0.78}} & \cellcolor[HTML]{FA7B1F}{\textcolor{white}{0.56}}\\
\hline
Sex & 10 & 19 & 1 & nominal & \cellcolor[HTML]{30123B}{\textcolor{white}{1.00}} & \cellcolor[HTML]{30123B}{\textcolor{white}{1.00}} & \cellcolor[HTML]{30123B}{\textcolor{white}{1.00}}\\
\hline
Schicksal & 10 & 19 & 2 & nominal & \cellcolor[HTML]{F5C53A}{\textcolor{black}{0.36}} & \cellcolor[HTML]{3CF58E}{\textcolor{black}{0.83}} & \cellcolor[HTML]{D9E436}{\textcolor{black}{0.66}}\\
\hline
Risiko & 10 & 19 & 4 & nominal & \cellcolor[HTML]{ED5510}{\textcolor{white}{0.20}} & \cellcolor[HTML]{AF1801}{\textcolor{white}{0.59}} & \cellcolor[HTML]{B21A01}{\textcolor{white}{0.46}}\\
\hline
Serviceinfos & 9 & 19 & 2 & nominal & \cellcolor[HTML]{DA3907}{\textcolor{white}{0.15}} & \cellcolor[HTML]{3CF58E}{\textcolor{black}{0.83}} & \cellcolor[HTML]{D9E436}{\textcolor{black}{0.66}}\\
\hline
Person & 10 & 19 & 5 & nominal & \cellcolor[HTML]{FE932A}{\textcolor{white}{0.28}} & \cellcolor[HTML]{E2430A}{\textcolor{black}{0.63}} & \cellcolor[HTML]{EC530F}{\textcolor{white}{0.53}}\\
\hline
Promis & 10 & 19 & 4 & nominal & \cellcolor[HTML]{D4E735}{\textcolor{black}{0.43}} & \cellcolor[HTML]{E7D739}{\textcolor{black}{0.73}} & \cellcolor[HTML]{EBD339}{\textcolor{black}{0.64}}\\
\hline
Emo & 10 & 19 & 4 & nominal & \cellcolor[HTML]{DA3907}{\textcolor{white}{0.15}} & \cellcolor[HTML]{FE932A}{\textcolor{black}{0.68}} & \cellcolor[HTML]{FC8725}{\textcolor{white}{0.57}}\\
\hline
Einfluss & 10 & 19 & 4 & nominal & \cellcolor[HTML]{84FF51}{\textcolor{black}{0.54}} & \cellcolor[HTML]{65FD69}{\textcolor{black}{0.81}} & \cellcolor[HTML]{65FD69}{\textcolor{black}{0.75}}\\
\hline
Kontroverse & 10 & 19 & 4 & nominal & \cellcolor[HTML]{E5D938}{\textcolor{black}{0.40}} & \cellcolor[HTML]{92FF47}{\textcolor{black}{0.79}} & \cellcolor[HTML]{96FE44}{\textcolor{black}{0.72}}\\
\hline
Konflikt & 10 & 19 & 2 & nominal & \cellcolor[HTML]{BE2102}{\textcolor{white}{0.10}} & \cellcolor[HTML]{3D9EFE}{\textcolor{white}{0.91}} & \cellcolor[HTML]{18DBC5}{\textcolor{black}{0.82}}\\
\hline
Normbruch & 10 & 19 & 4 & nominal & \cellcolor[HTML]{18E0BD}{\textcolor{black}{0.68}} & \cellcolor[HTML]{1AD4D0}{\textcolor{black}{0.87}} & \cellcolor[HTML]{18DBC5}{\textcolor{black}{0.82}}\\
\hline
Katastrophe & 10 & 19 & 4 & nominal & \cellcolor[HTML]{E9D539}{\textcolor{black}{0.39}} & \cellcolor[HTML]{65FD69}{\textcolor{black}{0.81}} & \cellcolor[HTML]{75FE5C}{\textcolor{black}{0.74}}\\
\hline
Aktual & 10 & 19 & 4 & nominal & \cellcolor[HTML]{A91601}{\textcolor{white}{0.07}} & \cellcolor[HTML]{7A0403}{\textcolor{white}{0.56}} & \cellcolor[HTML]{7A0403}{\textcolor{white}{0.42}}\\
\hline
Kontinuität & 10 & 19 & 2 & nominal & \cellcolor[HTML]{ACFB38}{\textcolor{black}{0.49}} & \cellcolor[HTML]{4EF97D}{\textcolor{black}{0.82}} & \cellcolor[HTML]{EBD339}{\textcolor{black}{0.64}}\\
\hline
Ueberrasch & 10 & 19 & 3 & nominal & \cellcolor[HTML]{7A0403}{\textcolor{white}{0.01}} & \cellcolor[HTML]{DDE037}{\textcolor{black}{0.74}} & \cellcolor[HTML]{FBB838}{\textcolor{black}{0.61}}\\
\hline
Sensation & 9 & 19 & 2 & nominal & \cellcolor[HTML]{F8721C}{\textcolor{white}{0.24}} & \cellcolor[HTML]{DDE037}{\textcolor{black}{0.74}} & \cellcolor[HTML]{BC2002}{\textcolor{white}{0.47}}\\
\hline
Superlative & 10 & 19 & 2 & nominal & \cellcolor[HTML]{7A0403}{\textcolor{white}{0.01}} & \cellcolor[HTML]{4664DA}{\textcolor{white}{0.95}} & \cellcolor[HTML]{4680F6}{\textcolor{white}{0.91}}\\
\hline
Kurios & 10 & 19 & 2 & nominal & \cellcolor[HTML]{F15D13}{\textcolor{white}{0.21}} & \cellcolor[HTML]{92FF47}{\textcolor{black}{0.79}} & \cellcolor[HTML]{FEA130}{\textcolor{white}{0.59}}\\
\hline
\end{tabular}
\end{table}

\begin{Shaded}
\begin{Highlighting}[]

\NormalTok{writexl}\SpecialCharTok{::}\FunctionTok{write\_xlsx}\NormalTok{(RelitestOutput, }\StringTok{"RelitestOutput.xlsx"}\NormalTok{)}
\end{Highlighting}
\end{Shaded}

\bookmarksetup{startatroot}

\hypertarget{daten-laden-und-fusionieren}{%
\chapter{Daten laden und
fusionieren}\label{daten-laden-und-fusionieren}}

\hypertarget{daten-aus-soscisurvey-tf-b}{%
\section{Daten aus SosciSurvey (TF
B)}\label{daten-aus-soscisurvey-tf-b}}

Gehen Sie in SosciSurvey unter \enquote{Steuerung} auf \enquote{Erhobene
Daten} und dann auf \enquote{Auswahlkritierien für gültige Fälle} und
haken Sie \enquote{Interview (Aufruf der Fragenbogen-URL)} an und den
Rest wieder ab, falls das noch an ist. Dann gehen Sie unter
\enquote{Erhobene Daten} auf \enquote{Daten herunterladen}. Dort finden
Sie mehrere Reiter für verschiedene Datentypen. Sie können Excel nehmen
und sich so die Daten bequemer ansehen. Nehmen Sie dann für den
R-Datensatz besser den Reiter \enquote{GNU R} und benennen Sie dort
unter \enquote{Name d.~Daten-Frame in R:} ihren Datensatz so wie Sie ihn
gerne hätten (meine Empfehlung DATEN\_BF). Bei \enquote{Variablen-Typen}
nehmen Sie \enquote{Nummerische Codes für Skalen \ldots{}} und bei
\enquote{Residualoptionen:} nehmen Sie \enquote{Werte behalten,
\ldots{}}. Etwas weiter unten sehen Sie dann noch den Reiter
\enquote{Variablen}. Dort nehmen Sie \enquote{Variablen, die im
heruntergeladenen Datensatz min. zwei unterschiedliche Ausprägungen
haben} (alles andere sind langweilige Konstanten). Darunter können Sie
noch die \enquote{Verweildauer \ldots{}} anhakeln und \enquote{Kennwerte
zur Datenqualität \ldots{}}. In der R-Onlineanleitung zu Methoden-Aufbau
haben Sie das auch im Video erklärt.

Einige Variablen kommen aus SosciSurvey als logische Variablen
(True/Fals), die wollen wir alle gleich mal in Dummys (0/1-Variablen)
umwandeln, weil wir wissen, dass 0 immer \enquote{FALSE} bedeutet und 1
\enquote{TRUE} und wir mit 0/1-Variablen (den Dummys) viel besser
arbeiten und rechnen können.

\begin{Shaded}
\begin{Highlighting}[]
\CommentTok{\# mutiere die Variablen für die gilt: is.logical als numerisch}
\NormalTok{DATEN\_BF }\OtherTok{\textless{}{-}}\NormalTok{ DATEN\_BF }\SpecialCharTok{|\textgreater{}} 
  \FunctionTok{mutate}\NormalTok{(}\FunctionTok{across}\NormalTok{(}\FunctionTok{where}\NormalTok{(is.logical), as.numeric))}
\end{Highlighting}
\end{Shaded}

\hypertarget{datenfusionen-tf-e}{%
\section{Datenfusionen (TF E)}\label{datenfusionen-tf-e}}

Da wir Kommunikationswissenschaftler sind und Medieneigenschaften mit
Befragungsergebnissen verbinden wollen, fusionieren wir hier die
Datensätze der Inhaltsanalyse mit denen der Befragung.

\hypertarget{die-befragungsdaten-an-die-inhaltsanalyse-matchen}{%
\subsection{Die Befragungsdaten an die Inhaltsanalyse
matchen}\label{die-befragungsdaten-an-die-inhaltsanalyse-matchen}}

\begin{Shaded}
\begin{Highlighting}[]

\CommentTok{\# Erstmal die beiden Datensätze laden:}

\NormalTok{IA }\OtherTok{\textless{}{-}}\NormalTok{ readxl}\SpecialCharTok{::}\FunctionTok{read\_excel}\NormalTok{(}\StringTok{"Daten/IA\_Gesamt.xlsx"}\NormalTok{)}
\NormalTok{BEF }\OtherTok{\textless{}{-}}\NormalTok{ readxl}\SpecialCharTok{::}\FunctionTok{read\_excel}\NormalTok{(}\StringTok{"Daten/Befragung ÜX.xlsx"}\NormalTok{)}

\CommentTok{\# Hier mal eine einfache Häufigkeitsauszählung}
\NormalTok{BEF }\SpecialCharTok{|\textgreater{}}\NormalTok{ sjmisc}\SpecialCharTok{::}\FunctionTok{frq}\NormalTok{(Nenn\_Code1, Nenn\_Code2, NennCode3)}
  
\CommentTok{\# Und hier wird gezählt, wie oft in BEF die verschiedenen Themen in der ersten Nennung vorkamen:}
\NormalTok{Nenn1\_DT }\OtherTok{\textless{}{-}}\NormalTok{ BEF }\SpecialCharTok{|\textgreater{}} \CommentTok{\# Erstelle eine neue Datentabelle (DT) "Nenn1" die aus BEF verarbeitet wurde und zwar wie folgt nach |\textgreater{} }
  \FunctionTok{count}\NormalTok{(Nenn\_Code1, }\AttributeTok{name =} \StringTok{"Nenn1"}\NormalTok{) }\SpecialCharTok{|\textgreater{}} \CommentTok{\# gruppiere nach der Variablen "Nenn\_Code1" im Datensatz und zähle die Häufigkeit und schreibe die in "Nenn1"}
  \FunctionTok{rename}\NormalTok{(}\StringTok{"Thema"} \OtherTok{=} \StringTok{"Nenn\_Code1"}\NormalTok{) }\CommentTok{\# Nenne jetzt noch die Bezeichnung "Nenn\_Code1" in "Thema" um, damit es nachher einfach mit "Thema" gematcht werden kann}

\CommentTok{\# ... und hier für die zweite}
\NormalTok{Nenn2\_DT }\OtherTok{\textless{}{-}}\NormalTok{ BEF }\SpecialCharTok{|\textgreater{}} 
  \FunctionTok{count}\NormalTok{(Nenn\_Code2, }\AttributeTok{name =} \StringTok{"Nenn2"}\NormalTok{) }\SpecialCharTok{|\textgreater{}} 
  \FunctionTok{rename}\NormalTok{(}\StringTok{"Thema"} \OtherTok{=} \StringTok{"Nenn\_Code2"}\NormalTok{)}

\CommentTok{\# ... Sie ahnen es:}
\NormalTok{Nenn3\_DT }\OtherTok{\textless{}{-}}\NormalTok{ BEF }\SpecialCharTok{|\textgreater{}} 
  \FunctionTok{count}\NormalTok{(NennCode3, }\AttributeTok{name =} \StringTok{"Nenn3"}\NormalTok{) }\SpecialCharTok{|\textgreater{}} 
  \FunctionTok{rename}\NormalTok{(}\StringTok{"Thema"} \OtherTok{=} \StringTok{"NennCode3"}\NormalTok{)}

\CommentTok{\# Jetzt bauen wir einen Datensatz für die Themen, die Zählungen hat, wenn sie in "Nenn1\_DT" bis "Nenn3\_DT" genannt wurden:}
\NormalTok{Nennungen }\OtherTok{\textless{}{-}} \FunctionTok{full\_join}\NormalTok{(Nenn1\_DT, Nenn2\_DT, }\AttributeTok{by =} \StringTok{"Thema"}\NormalTok{) }\SpecialCharTok{|\textgreater{}} \CommentTok{\# erstmal die Nenn2{-}Daten an die Nenn2{-}Daten}
  \FunctionTok{full\_join}\NormalTok{(., Nenn3\_DT, }\AttributeTok{by =} \StringTok{"Thema"}\NormalTok{) }\SpecialCharTok{|\textgreater{}} \CommentTok{\# dann noch die Nenn3\_DT an die beiden von oben}
  \FunctionTok{rowwise}\NormalTok{() }\SpecialCharTok{|\textgreater{}} \CommentTok{\# mache einen Summenindex, der Zeilenweise (rowwise) ...}
  \FunctionTok{mutate}\NormalTok{(}\AttributeTok{Nenn\_Gesamt =} \FunctionTok{sum}\NormalTok{(}\FunctionTok{c}\NormalTok{(Nenn1, Nenn2, Nenn3))) }\CommentTok{\# ... die Anzahl der Nennungen aus Nenn1 bis Nenn3 aufsummiert = Anzahl der Themanennungen}

\CommentTok{\# Diese Anzahl der Nennungen kleben wir jetzt hinten an den IA{-}Datensatz und nennen den fusionierten Datensatz "IAuBEF"}
\NormalTok{IAuBEF }\OtherTok{\textless{}{-}} \FunctionTok{left\_join}\NormalTok{(IA, Nennungen, }\AttributeTok{by =} \FunctionTok{c}\NormalTok{(}\StringTok{"Thema"} \OtherTok{=} \StringTok{"Thema"}\NormalTok{))}
\end{Highlighting}
\end{Shaded}

\hypertarget{IAanBefragung}{%
\subsection{Die Inhaltsanylsedaten an die Befragung
machten}\label{IAanBefragung}}

\begin{Shaded}
\begin{Highlighting}[]

\CommentTok{\# Hier mache ich eine Tabelle "IA{-}Thema", }
\NormalTok{IA\_Thema }\OtherTok{\textless{}{-}}\NormalTok{ DATEN\_IA }\SpecialCharTok{|\textgreater{}} 
  \FunctionTok{filter}\NormalTok{(}\SpecialCharTok{!}\FunctionTok{is.na}\NormalTok{(TC)) }\SpecialCharTok{|\textgreater{}} \CommentTok{\# die keine NAs enthält und ...}
  \FunctionTok{group\_by}\NormalTok{(TC) }\SpecialCharTok{|\textgreater{}} \CommentTok{\#... und nach "Thema" gruppiert ist, also Infos pro Thema enthält und zwar ...}
  \FunctionTok{summarise}\NormalTok{(}\AttributeTok{MedienVorkomm =} \FunctionTok{n}\NormalTok{(), }\CommentTok{\# Die Anzahl der Artikel, wo das Thema vorkam}
            \AttributeTok{KURIO =} \FunctionTok{mean}\NormalTok{(KURIO),   }\CommentTok{\# Die durchschnittliche Reichweite, die für die Artikel zu diesem Thema kodiert wurden}
            \AttributeTok{KNTR =} \FunctionTok{mean}\NormalTok{(KNTR),   }\CommentTok{\# das Gleiche für das Risiko}
            \AttributeTok{SCHICK =} \FunctionTok{mean}\NormalTok{(SCHICK), }
\NormalTok{            )   }\CommentTok{\# und Personalisierung }

\CommentTok{\# Hier werden jetzt, ähnlich wie oben, die Datensätze der IA an die BEF geklebt und zu BEFuIA fusioniert}
\NormalTok{BEFuIA }\OtherTok{\textless{}{-}}\NormalTok{ DATEN\_BF }\SpecialCharTok{|\textgreater{}} 
  \FunctionTok{select}\NormalTok{(}\SpecialCharTok{{-}}\FunctionTok{any\_of}\NormalTok{(}\FunctionTok{c}\NormalTok{(}\StringTok{"MedienVorkommen1"}\NormalTok{, }\StringTok{"KURIO1"}\NormalTok{, }\StringTok{"KNTR1"}\NormalTok{, }\StringTok{"SCHICK1"}\NormalTok{, }
                 \StringTok{"MedienVorkommen2"}\NormalTok{, }\StringTok{"KURIO2"}\NormalTok{, }\StringTok{"KNTR2"}\NormalTok{, }\StringTok{"SCHICK2"}\NormalTok{, }
                 \StringTok{"MedienVorkommen3"}\NormalTok{, }\StringTok{"KURIO3"}\NormalTok{, }\StringTok{"KNTR3"}\NormalTok{, }\StringTok{"SCHICK3"}\NormalTok{))) }\SpecialCharTok{|\textgreater{}} 
  \FunctionTok{left\_join}\NormalTok{(., IA\_Thema, }\AttributeTok{by =} \FunctionTok{c}\NormalTok{(}\StringTok{"TCW1"} \OtherTok{=} \StringTok{"TC"}\NormalTok{)) }\SpecialCharTok{|\textgreater{}} \CommentTok{\# zu jeder Nennung (TCW) wir das "Thema" und die Mittelwerte der Variablen gehängt}
  \FunctionTok{rename}\NormalTok{(}\AttributeTok{MedienVorkommen1 =} \StringTok{"MedienVorkomm"}\NormalTok{, }\AttributeTok{KURIO1 =} \StringTok{"KURIO"}\NormalTok{, }\AttributeTok{KNTR1 =} \StringTok{"KNTR"}\NormalTok{, }\AttributeTok{SCHICK1 =} \StringTok{"SCHICK"}\NormalTok{) }\SpecialCharTok{|\textgreater{}} \CommentTok{\# hier umbenennen, damit es die Variablen nicht doppelt gibt}
  \FunctionTok{left\_join}\NormalTok{(., IA\_Thema, }\AttributeTok{by =} \FunctionTok{c}\NormalTok{(}\StringTok{"TCW2"} \OtherTok{=} \StringTok{"TC"}\NormalTok{)) }\SpecialCharTok{|\textgreater{}} \CommentTok{\# jetzt die Themenvariablen nach Nenn2}
  \FunctionTok{rename}\NormalTok{(}\AttributeTok{MedienVorkommen2 =} \StringTok{"MedienVorkomm"}\NormalTok{, }\AttributeTok{KURIO2 =} \StringTok{"KURIO"}\NormalTok{, }\AttributeTok{KNTR2 =} \StringTok{"KNTR"}\NormalTok{, }\AttributeTok{SCHICK2 =} \StringTok{"SCHICK"}\NormalTok{) }\SpecialCharTok{|\textgreater{}}  \CommentTok{\# wieder umbenennen}
  \FunctionTok{left\_join}\NormalTok{(., IA\_Thema, }\AttributeTok{by =} \FunctionTok{c}\NormalTok{(}\StringTok{"TCW3"} \OtherTok{=} \StringTok{"TC"}\NormalTok{)) }\SpecialCharTok{|\textgreater{}} \CommentTok{\# und, is klar, gell}
  \FunctionTok{rename}\NormalTok{(}\AttributeTok{MedienVorkommen3 =} \StringTok{"MedienVorkomm"}\NormalTok{, }\AttributeTok{KURIO3 =} \StringTok{"KURIO"}\NormalTok{, }\AttributeTok{KNTR3 =} \StringTok{"KNTR"}\NormalTok{, }\AttributeTok{SCHICK3 =} \StringTok{"SCHICK"}\NormalTok{) }\CommentTok{\# und umbennenen, damit wir auch hier wissen, dass es zur Nenn3 gehört}

\NormalTok{DATEN\_BF }\OtherTok{\textless{}{-}}\NormalTok{ BEFuIA }\SpecialCharTok{|\textgreater{}} \CommentTok{\# Jetzt machte ich aus den Variablen Themen noch jeweils einen Mittelwert als Interesse an den NFs}
  \FunctionTok{rowwise}\NormalTok{() }\SpecialCharTok{|\textgreater{}} 
  \FunctionTok{mutate}\NormalTok{(}\AttributeTok{KURIO\_m =} \FunctionTok{mean}\NormalTok{(}\FunctionTok{c}\NormalTok{(KURIO1, KURIO2, KURIO3), }\AttributeTok{na.rm =}\NormalTok{ T)) }\SpecialCharTok{|\textgreater{}} 
  \FunctionTok{mutate}\NormalTok{(}\AttributeTok{KNTR\_m =} \FunctionTok{mean}\NormalTok{(}\FunctionTok{c}\NormalTok{(KNTR1, KNTR2, KNTR3), }\AttributeTok{na.rm =}\NormalTok{ T)) }\SpecialCharTok{|\textgreater{}} 
  \FunctionTok{mutate}\NormalTok{(}\AttributeTok{SCHICK\_m =} \FunctionTok{mean}\NormalTok{(}\FunctionTok{c}\NormalTok{(SCHICK1, SCHICK2, SCHICK3), }\AttributeTok{na.rm =}\NormalTok{ T)) }\SpecialCharTok{|\textgreater{}} 
  \FunctionTok{mutate}\NormalTok{(}\AttributeTok{KURIO\_m =} \FunctionTok{replace}\NormalTok{(KURIO\_m, }\FunctionTok{is.na}\NormalTok{(KURIO\_m), }\DecValTok{0}\NormalTok{)) }\SpecialCharTok{|\textgreater{}} \CommentTok{\# für eine Variable die NA durch 0 ersetzen (kopieren)}
  \FunctionTok{mutate}\NormalTok{(}\FunctionTok{across}\NormalTok{(}\FunctionTok{c}\NormalTok{(KNTR\_m, KURIO\_m, SCHICK\_m), }\SpecialCharTok{\textasciitilde{}}\FunctionTok{replace}\NormalTok{(.x, }\FunctionTok{is.na}\NormalTok{(.x), }\DecValTok{0}\NormalTok{) )) }\CommentTok{\# across mehrere Varialblen gleichzeitig NA durch 0}

\FunctionTok{saveRDS}\NormalTok{(DATEN\_BF, }\AttributeTok{file =} \StringTok{"Daten/DATEN\_BF.RDS"}\NormalTok{)}
\end{Highlighting}
\end{Shaded}

\hypertarget{daten-speichern-und-laden-tf-e}{%
\section{Daten speichern und laden (TF
E)}\label{daten-speichern-und-laden-tf-e}}

Wenn Sie durch den Import der Daten aus SosciSurvey ein Objekt mit dem
Namen DATEN\_BF haben, dann können Sie dieses Objekt als R-Daten
speichern. Später können Sie diese Daten immer wieder mit readRDS
aufrufen. Sie können so auch zwischendurch mal ihre Daten als Datensatz
speichern. Ich selbst packe mir diese Zeilen immer ganz an den Anfang in
den r-Chung \enquote{Setup} wo auch die Pakete geladen werden. Die Daten
liegen und werden abgespeichert im Unterordner \enquote{Daten}.

\begin{Shaded}
\begin{Highlighting}[]
\NormalTok{DATEN\_BF }\OtherTok{\textless{}{-}} \FunctionTok{readRDS}\NormalTok{(}\StringTok{"Daten/DATEN\_BF.RDS"}\NormalTok{)}
\FunctionTok{saveRDS}\NormalTok{(DATEN\_BF, }\AttributeTok{file =} \StringTok{"Daten/DATEN\_BF.RDS"}\NormalTok{)}

\CommentTok{\# Das erste Mal oder wenn Sie nochmal was in der Excel editiert haben:}
\CommentTok{\#DATEN\_IA \textless{}{-} read\_xlsx("Daten/DATEN\_IA.xlsx")}

\NormalTok{DATEN\_IA }\OtherTok{\textless{}{-}} \FunctionTok{readRDS}\NormalTok{(}\StringTok{"Daten/DATEN\_IA.RDS"}\NormalTok{)}
\FunctionTok{saveRDS}\NormalTok{(DATEN\_IA, }\AttributeTok{file =} \StringTok{"Daten/DATEN\_IA.RDS"}\NormalTok{)}

\NormalTok{DATEN\_BF }\OtherTok{\textless{}{-}} \FunctionTok{readRDS}\NormalTok{(}\StringTok{"Daten/BEFuIA.RDS"}\NormalTok{)}

\NormalTok{DATEN\_IA }\OtherTok{\textless{}{-}} \FunctionTok{readRDS}\NormalTok{(}\StringTok{"Daten/IAuBEF.RDS"}\NormalTok{)}
\end{Highlighting}
\end{Shaded}

\bookmarksetup{startatroot}

\hypertarget{datenaufbereitung}{%
\chapter{Datenaufbereitung}\label{datenaufbereitung}}

Laden Sie sich als erstes die Datei Auswertung.Rmd herunter. Darin
finden Sie Codebeispiele, die Sie kopieren können und so ändern, dass
Sie auf Ihre Datensätze und zu Ihren Variablen passen.

\begin{Shaded}
\begin{Highlighting}[]
\NormalTok{ vembedr}\SpecialCharTok{::}\FunctionTok{embed\_youtube}\NormalTok{(}\StringTok{"huBJnXr70SE"}\NormalTok{) }\SpecialCharTok{|\textgreater{}} 
\NormalTok{  vembedr}\SpecialCharTok{::}\FunctionTok{use\_rounded}\NormalTok{()}
\end{Highlighting}
\end{Shaded}

\hypertarget{leere-fuxe4lle-luxf6schen-tf-e}{%
\section{Leere Fälle löschen (TF
E)}\label{leere-fuxe4lle-luxf6schen-tf-e}}

Hier gehe ich davon aus, dass ein Interview erfolgreich beendet wurde,
wenn der:die Interviewer:in am Ende etwas zum Verständnis des Interviews
eingegeben hat.

\begin{Shaded}
\begin{Highlighting}[]
\NormalTok{DATEN\_BF }\OtherTok{\textless{}{-}}\NormalTok{ DATEN\_BF }\SpecialCharTok{|\textgreater{}} 
  \FunctionTok{filter}\NormalTok{(}\SpecialCharTok{!}\FunctionTok{is.na}\NormalTok{(PLZ))}

\DocumentationTok{\#\# Das geht nur einmal, weil R dann die Variablen nicht mehr findet, die wir gerade gelöscht haben, logisch}
\NormalTok{DATEN\_BF }\OtherTok{\textless{}{-}}\NormalTok{ DATEN\_BF }\SpecialCharTok{|\textgreater{}} 
  \FunctionTok{select}\NormalTok{(}\SpecialCharTok{{-}}\FunctionTok{c}\NormalTok{(SERIAL}\SpecialCharTok{:}\NormalTok{MODE)) }

\NormalTok{DATEN\_BF }\OtherTok{\textless{}{-}}\NormalTok{ DATEN\_BF }\SpecialCharTok{|\textgreater{}} 
  \FunctionTok{relocate}\NormalTok{(INTERVIEWER, }\AttributeTok{.after =}\NormalTok{ STARTED)}
\end{Highlighting}
\end{Shaded}

\hypertarget{variablen-umbenennen-und-luxf6schen}{%
\section{Variablen umbenennen und
löschen}\label{variablen-umbenennen-und-luxf6schen}}

Wenn Sie den einfachen Befehl \enquote{select} nehmen, dann werden die
Variablen behalten, die Sie aufführen. Wenn Sie ein Minus davor
schreiben, werden die Variablen gelöscht, die Sie aufführen. Häufig
führt das später zu Fehlermeldungen, weil bei mehrfacher Ausführung
gelöschte Variablen natürlich nicht mehr im Datensatz gefunden werden.
Um diesen Fehlermeldungen aus dem Weg zu gehen, können Sie
\enquote{any\_of} verwenden. Dann prüft R, ob eine der aufgeführten
Variablen im Datensatz sind und löscht diese. Wenn die Variablen nicht
im Datensatz sind, dann macht R nichts weiter (meldet auch keinen Fehler
und stoppt Ihren Durchlauf nicht.). Seien Sie damit also mir Vorsicht.
Beim Befehl \enquote{rename} können Sie das auch machen. Mit
\enquote{any\_of} werden nur die Variablen umbenannt, die auch wirklich
da sind. Manchmal kommt das bei jedem Durchlauf vor, weil zB bei
join-Befehlen immer nochmal ein Datensatz gematcht wird und wie im
Beispiel vorhandene Variablen wie \enquote{Nenn1} umbenannt werden in
\enquote{Nenn1.x} und Nenn1.y'' Wenn Sie jetzt die mit \enquote{.y}
löschen und die mit \enquote{.x} umbenennen, dann haben Sie das Problem
auch glöst, wenn Ihnen nicht ganz klar ist, warum die immerwieder
auftauchen. Für's Erste ist das also auch eine Lösung eines Problems.
;-)

\begin{Shaded}
\begin{Highlighting}[]
\CommentTok{\# Manchmal macht es Sinn, von einer Variable eine Kopie mit besser lesbarem Namen anzulegen, wie hier:}

\NormalTok{DATEN\_BF }\OtherTok{\textless{}{-}}\NormalTok{ DATEN\_BF }\SpecialCharTok{|\textgreater{}} 
  \FunctionTok{mutate}\NormalTok{(}\AttributeTok{Alter =}\NormalTok{ SD01,}
         \AttributeTok{Geschlecht =}\NormalTok{ SD02)}

\CommentTok{\# So können Sie Variablen löschen. Durch das {-}any\_of gibt es keine Fehler, wenn R die Variablen nicht findet}
\NormalTok{DATEN\_BF }\OtherTok{\textless{}{-}}\NormalTok{ DATEN\_BF }\SpecialCharTok{|\textgreater{}} 
  \FunctionTok{select}\NormalTok{(}\SpecialCharTok{{-}}\FunctionTok{any\_of}\NormalTok{(}\FunctionTok{c}\NormalTok{(}\StringTok{"Nenn1.y"}\NormalTok{, }\StringTok{"Nenn2.y"}\NormalTok{, }\StringTok{"Nann3.y"}\NormalTok{)))}
\CommentTok{\# So nenne Sie Variablen um (vorne wie es bisher hiess).}

\NormalTok{DATEN\_BF }\OtherTok{\textless{}{-}}\NormalTok{ DATEN\_BF }\SpecialCharTok{|\textgreater{}} 
  \FunctionTok{rename}\NormalTok{(}\FunctionTok{any\_of}\NormalTok{(}\FunctionTok{c}\NormalTok{(}\StringTok{"Nenn1.x"} \OtherTok{=} \StringTok{"Nenn1"}\NormalTok{, }\StringTok{"Nenn2.x"} \OtherTok{=} \StringTok{"Nenn2"}\NormalTok{, }\StringTok{"Nenn3.x"} \OtherTok{=} \StringTok{"Nenn3"}\NormalTok{))) }
\end{Highlighting}
\end{Shaded}

\hypertarget{label-fuxfcr-variablen-und-auspruxe4gungen-teils-tf-e}{%
\section{Label für Variablen und Ausprägungen (teils TF
E)}\label{label-fuxfcr-variablen-und-auspruxe4gungen-teils-tf-e}}

Wenn Sie messen, dann stehen am Ende Zahlen für das was Sie gemessen,
also zB erfragt haben. Zum Beispiel steht dann eine 1 für
\enquote{trifft nicht zu} und 5 für \enquote{trifft vollkommen zu}. In
Ihren Auswertungen müssen Sie aber wieder die verbalen Entsprechungen
nutze, statt nur Tabellen oder Grafiken zu haben, wo nur Zahlen
ausgewiesen werden (Sie können das noch verbinden, aber Ihre Leser:innen
nicht). In anderen Programmen (wie SPSS) ist es daher üblich, dass die
gemessenen Zahlen mit verbalen Labeln versehen werden. In Base-R geht
das nicht so toll. Da müssen Sie mit \enquote{factors} arbeiten, wo also
statt der Zahlen die verbalen Entsprechungen verwendet werden. Mühsam an
den Faktoren ist, dass man mit ihnen wieder nicht mehr rechnen kann,
also z.B. weder einen Mittelwert ausgeben noch eine Korrelation. Da
hilft das Paket sjmisc weiter. Mit dem können Sie den Zahlenwerten Ihrer
Variablen ihre Merkmalsausprägungen als Label zuordnen. Das geht mit
set\_label(labels = c(\enquote{tzrifft nicht zu} = 1, \enquote{trifft
voll zu} = 5)):

\begin{Shaded}
\begin{Highlighting}[]
\CommentTok{\# So können Label vergeben werden}

\NormalTok{DATEN\_BF }\SpecialCharTok{|\textgreater{}}\NormalTok{ sjmisc}\SpecialCharTok{::}\FunctionTok{frq}\NormalTok{(GESCHL)}

\CommentTok{\# Hier wird das Variablenlabel vergeben}

\NormalTok{DATEN\_BF }\OtherTok{\textless{}{-}}\NormalTok{ DATEN\_BF }\SpecialCharTok{|\textgreater{}} 
\NormalTok{  sjlabelled}\SpecialCharTok{::}\FunctionTok{var\_labels}\NormalTok{(}\AttributeTok{GESCHL =} \StringTok{"Geschlecht"}\NormalTok{) }

\CommentTok{\# Hier werden für die einzelnen Ausprägungen die Label vergeben}
\NormalTok{DATEN\_BF }\OtherTok{\textless{}{-}}\NormalTok{ DATEN\_BF }\SpecialCharTok{|\textgreater{}} 
\NormalTok{  sjlabelled}\SpecialCharTok{::}\FunctionTok{set\_labels}\NormalTok{(GESCHL, }\AttributeTok{labels =} \FunctionTok{c}\NormalTok{(}\StringTok{"mänlich"} \OtherTok{=} \DecValTok{1}\NormalTok{, }\StringTok{"weiblich"} \OtherTok{=} \DecValTok{2}\NormalTok{, }\StringTok{"divers"} \OtherTok{=} \DecValTok{3}\NormalTok{, }\StringTok{"k.A."} \OtherTok{=} \ConstantTok{NA}\NormalTok{))}

\CommentTok{\# Mit dem Befehl für Häufigkeitstabellen (frq für frequencies) kann die Labelei schnell angesehen werden:}

\NormalTok{DATEN\_BF }\SpecialCharTok{|\textgreater{}} 
\NormalTok{  sjmisc}\SpecialCharTok{::}\FunctionTok{frq}\NormalTok{(GESCHL)}

\DocumentationTok{\#\# Hier werden für einige Variablen Label vergeben, die zusammengehören und später als Auswertung für Mehrfrachantworten verarbeitet werden}

\CommentTok{\# Ersteinmal die Variablen angucken. Die Variablen gehen von NACHRICHT\_TITEL1 in einer Reihe bis M112\_23 (wurde nicht umbenannt in SosciSurvey)}
\NormalTok{DATEN\_BF }\SpecialCharTok{|\textgreater{}} 
  \FunctionTok{select}\NormalTok{(NACHRICHT\_TITEL1}\SpecialCharTok{:}\NormalTok{M112\_23) }\SpecialCharTok{|\textgreater{}} \CommentTok{\# Nur die interessierenden Variablen selektieren}
  \FunctionTok{multiResponse}\NormalTok{() }\SpecialCharTok{|\textgreater{}} \CommentTok{\# Mehrfachantworten ansehen}
  \FunctionTok{arrange}\NormalTok{(}\FunctionTok{desc}\NormalTok{(Frequency)) }\SpecialCharTok{|\textgreater{}} \CommentTok{\# Sortieren nach Häufigkeit}
\NormalTok{  kableExtra}\SpecialCharTok{::}\FunctionTok{kable}\NormalTok{() }\SpecialCharTok{|\textgreater{}} \CommentTok{\# besser lesbare Tabelle daraus machen}
\NormalTok{  kableExtra}\SpecialCharTok{::}\FunctionTok{kable\_styling}\NormalTok{() }\CommentTok{\# Styling}


\CommentTok{\# Mit set\_variable\_labels werden hier die Variablenlabel vergeben}
\NormalTok{DATEN\_BF }\OtherTok{\textless{}{-}}\NormalTok{ DATEN\_BF }\SpecialCharTok{|\textgreater{}}
\NormalTok{  sjlabelled}\SpecialCharTok{::}\FunctionTok{var\_labels}\NormalTok{(}\AttributeTok{NACHRICHT\_TITEL1 =} \StringTok{"NZZ"}\NormalTok{, }
                      \AttributeTok{NACHRICHT\_TITEL2 =} \StringTok{"TA"}\NormalTok{, }
                      \AttributeTok{NACHRICHT\_TITEL3 =} \StringTok{"20 Minuten"}\NormalTok{,}
                      \AttributeTok{NACHRICHT\_TITEL4 =} \StringTok{"Blick"}\NormalTok{, }
                      \AttributeTok{NACHRICHT\_TITEL5 =} \StringTok{"Watson"}\NormalTok{,}
                      \AttributeTok{NACHRICHT\_TITEL6 =} \StringTok{"SRF News"}\NormalTok{, }
                      \AttributeTok{M112\_20 =} \StringTok{"Social Media"}\NormalTok{,}
                      \AttributeTok{M112\_21 =} \StringTok{"Google News"}\NormalTok{,}
                      \AttributeTok{M112\_23 =} \StringTok{"anderes"}
\NormalTok{                      )}

\CommentTok{\# Hier mal für den Nachrichtenfaktor "Kontroverse" aus der Inhaltsanalyse}

\NormalTok{DATEN\_IA }\OtherTok{\textless{}{-}}\NormalTok{ DATEN\_IA }\SpecialCharTok{|\textgreater{}} 
\NormalTok{ sjlabelled}\SpecialCharTok{::}\FunctionTok{var\_labels}\NormalTok{(KNTR, }\AttributeTok{label =} \StringTok{"Kontroverse"}\NormalTok{) }\SpecialCharTok{|\textgreater{}} 
\NormalTok{ sjlabelled}\SpecialCharTok{::}\FunctionTok{set\_labels}\NormalTok{(KNTR, }\AttributeTok{labels =} \FunctionTok{c}\NormalTok{(}\StringTok{"keine"} \OtherTok{=} \DecValTok{0}\NormalTok{, }\StringTok{"gering"} \OtherTok{=} \DecValTok{1}\NormalTok{, }\StringTok{"stark"} \OtherTok{=} \DecValTok{2}\NormalTok{, }\StringTok{"k.A."} \OtherTok{=} \ConstantTok{NA}\NormalTok{))}
\end{Highlighting}
\end{Shaded}

\hypertarget{befragungstag-aus-dem-datum}{%
\section{Befragungstag aus dem
Datum}\label{befragungstag-aus-dem-datum}}

Mit diesem Befehl kann der Befragungstag aus der Datumsvariable erzeugt
werden, die von SosciSurvey automatisch erzeugt wird.

\begin{Shaded}
\begin{Highlighting}[]
\NormalTok{DATEN\_BF }\OtherTok{\textless{}{-}}\NormalTok{ DATEN\_BF }\SpecialCharTok{|\textgreater{}} 
  \FunctionTok{mutate}\NormalTok{(}\AttributeTok{Befrag\_Tag =} \FunctionTok{day}\NormalTok{(STARTED)) }\SpecialCharTok{|\textgreater{}} \CommentTok{\# mit day wird aus der Datumsvariable der Tag rausgezogen}
  \FunctionTok{relocate}\NormalTok{(Befrag\_Tag, }\AttributeTok{.after =}\NormalTok{ STARTED) }\CommentTok{\# Hiermit wird die Variable für den Befragungsstart an den Anfang gepackt}

\NormalTok{DATEN\_BF }\SpecialCharTok{|\textgreater{}}\NormalTok{ sjmisc}\SpecialCharTok{::}\FunctionTok{frq}\NormalTok{(Befrag\_Tag)}
\end{Highlighting}
\end{Shaded}

\hypertarget{umkodieren}{%
\section{Umkodieren}\label{umkodieren}}

\hypertarget{werte-wechseln}{%
\subsection{Werte wechseln}\label{werte-wechseln}}

Umkodieren bedeutet, dass Ausprägungen (also Werte) in einer Variablen
verändert werden oder zu anderen Werten in einer neuen Variable werden.
Typische Umkodierungen sind Veränderungen der Codes (also zB wenn man
eine Skala umdrehen möchte von 1 bis 5 auf 5 bis 1) und
Zusammenfassungen von Codes, also zB 1 und 2 werden neu in 1
zusammengefasst 3 wird die neue 2 und 4 sowie 5 werden als neue 3
kodiert. Dann hat man jeweils zwei Abstufungen in den Ausprägungen zu
jeweils einer zusammengefasst.

\begin{Shaded}
\begin{Highlighting}[]
\DocumentationTok{\#\# Hier machen wir aus den beiden tieferen Bildungsgruppen eine einzelne Gruppe. In den Klammern kann man die Value Label gleich mit vergeben:}

\NormalTok{DATEN\_BF }\SpecialCharTok{|\textgreater{}}\NormalTok{ sjmisc}\SpecialCharTok{::}\FunctionTok{frq}\NormalTok{(BILDUNG) }\CommentTok{\# erstmal angucken}

\CommentTok{\# Erstelle mit mutate eine neue Variable BILDUNG\_gr, die eine Rekodierung (rec) der Variable BILDUNG (zB 1 und 2 werden 1 mit dem Label "tief")}
\NormalTok{DATEN\_BF }\OtherTok{\textless{}{-}}\NormalTok{ DATEN\_BF }\SpecialCharTok{|\textgreater{}} 
 \FunctionTok{mutate}\NormalTok{(}\AttributeTok{BILDUNG\_gr =} \FunctionTok{rec}\NormalTok{(BILDUNG, }\AttributeTok{rec =} \StringTok{"1:2 = 1; 3 = 2; 4 = 3; 5 = 4"}\NormalTok{))  }\CommentTok{\# die offenen müssen noch kodiert werden}

\NormalTok{DATEN\_BF }\OtherTok{\textless{}{-}}\NormalTok{ DATEN\_BF }\SpecialCharTok{|\textgreater{}} 
\NormalTok{  sjlabelled}\SpecialCharTok{::}\FunctionTok{set\_labels}\NormalTok{(BILDUNG\_gr, }\AttributeTok{labels =} \FunctionTok{c}\NormalTok{(}\StringTok{"tief"} \OtherTok{=} \DecValTok{1}\NormalTok{, }\StringTok{"mittel"} \OtherTok{=} \DecValTok{2}\NormalTok{, }\StringTok{"hoch"} \OtherTok{=} \DecValTok{3}\NormalTok{, }\StringTok{"noch kodieren"} \OtherTok{=} \DecValTok{4}\NormalTok{), }\AttributeTok{force.labels =} \ConstantTok{TRUE}\NormalTok{) }\SpecialCharTok{|\textgreater{}} 
\NormalTok{  sjlabelled}\SpecialCharTok{::}\FunctionTok{var\_labels}\NormalTok{(BILDUNG\_gr, }\AttributeTok{label =} \StringTok{"Bildung (gruppiert)"}\NormalTok{)  }\CommentTok{\# hier das Variablenlabel}

\NormalTok{DATEN\_BF }\SpecialCharTok{|\textgreater{}} 
\NormalTok{    sjmisc}\SpecialCharTok{::}\FunctionTok{frq}\NormalTok{(BILDUNG\_gr)}
\end{Highlighting}
\end{Shaded}

\hypertarget{mehrere-auf-einmal-umkodieren}{%
\subsection{Mehrere auf einmal
umkodieren}\label{mehrere-auf-einmal-umkodieren}}

\begin{Shaded}
\begin{Highlighting}[]
\NormalTok{DATEN\_BF }\SpecialCharTok{|\textgreater{}} 
  \FunctionTok{frq}\NormalTok{(HAUFIGKEIT\_01}\SpecialCharTok{:}\NormalTok{HAUFIGKEIT\_05)}

\CommentTok{\# mutiere across die ganze Liste der Variablen von:bis, als Funktion (\textasciitilde{}) rec(.x) wobei .x für die Liste steht}
\NormalTok{DATEN\_BF }\OtherTok{\textless{}{-}}\NormalTok{ DATEN\_BF }\SpecialCharTok{|\textgreater{}} 
  \FunctionTok{mutate}\NormalTok{(}\FunctionTok{across}\NormalTok{(HAUFIGKEIT\_01}\SpecialCharTok{:}\NormalTok{HAUFIGKEIT\_05, }
                \SpecialCharTok{\textasciitilde{}} \FunctionTok{rec}\NormalTok{(.x, }\AttributeTok{rec =} \StringTok{"1 = {-}2 [viel zu wenig]; 2 = {-}1 [zu wenig]; 3 = 0 [richtig]; 4 = 1  [zu viel]; 5 = 2 [viel zu viel]; else = NA"}\NormalTok{), }
                       \AttributeTok{.names =} \StringTok{"\{.col\}\_um0"}\NormalTok{)) }\SpecialCharTok{|\textgreater{}} \CommentTok{\# HIer werden an die Variablennamen Suffixe "\_um0" angehängt}
\NormalTok{  sjlabelled}\SpecialCharTok{::}\FunctionTok{var\_labels}\NormalTok{(}\AttributeTok{HAUFIGKEIT\_01\_um0 =} \StringTok{"Corona"}\NormalTok{, }\CommentTok{\# Hier noch Label für die Variablen}
             \AttributeTok{HAUFIGKEIT\_02\_um0 =} \StringTok{"Abstimmungen"}\NormalTok{, }
             \AttributeTok{HAUFIGKEIT\_03\_um0 =} \StringTok{"Finanz"}\NormalTok{, }
             \AttributeTok{HAUFIGKEIT\_04\_um0 =} \StringTok{"Klima"}\NormalTok{, }
             \AttributeTok{HAUFIGKEIT\_05\_um0 =} \StringTok{"Geflüchtete"}\NormalTok{)}

\NormalTok{DATEN\_BF }\SpecialCharTok{|\textgreater{}} 
\NormalTok{  sjmisc}\SpecialCharTok{::}\FunctionTok{frq}\NormalTok{(HAUFIGKEIT\_01\_um0}\SpecialCharTok{:}\NormalTok{HAUFIGKEIT\_05\_um0)}
\end{Highlighting}
\end{Shaded}

\hypertarget{na-durch-0-ersetzen}{%
\subsection{NA durch 0 ersetzen}\label{na-durch-0-ersetzen}}

\begin{Shaded}
\begin{Highlighting}[]
\NormalTok{DATEN\_IA }\SpecialCharTok{|\textgreater{}} 
  \FunctionTok{mutate}\NormalTok{(}\FunctionTok{across}\NormalTok{(Nenn1}\SpecialCharTok{:}\NormalTok{Nenn5), }\FunctionTok{ifelse}\NormalTok{(}\FunctionTok{is.na}\NormalTok{(.x) }\SpecialCharTok{==} \ConstantTok{TRUE}\NormalTok{, }\DecValTok{0}\NormalTok{, .x))}\ErrorTok{)}
\end{Highlighting}
\end{Shaded}

\hypertarget{auspruxe4gungen-zusammenfassen}{%
\subsection{Ausprägungen
zusammenfassen}\label{auspruxe4gungen-zusammenfassen}}

\begin{Shaded}
\begin{Highlighting}[]
\CommentTok{\# Das Alter in Gruppen zusammenfassen}
\NormalTok{DATEN\_BF }\OtherTok{\textless{}{-}}\NormalTok{ DATEN\_BF }\SpecialCharTok{|\textgreater{}} 
  \FunctionTok{mutate}\NormalTok{(}\AttributeTok{ALTER\_gr =} \FunctionTok{case\_when}\NormalTok{(}
\NormalTok{    ALTER }\SpecialCharTok{\textless{}} \DecValTok{0} \SpecialCharTok{\textasciitilde{}} \ConstantTok{NA\_real\_}\NormalTok{,}
\NormalTok{    ALTER }\SpecialCharTok{\textless{}=} \DecValTok{25} \SpecialCharTok{\textasciitilde{}} \DecValTok{1}\NormalTok{,}
\NormalTok{    ALTER }\SpecialCharTok{\textless{}=} \DecValTok{45} \SpecialCharTok{\textasciitilde{}} \DecValTok{2}\NormalTok{,}
\NormalTok{    ALTER }\SpecialCharTok{\textless{}=} \DecValTok{65} \SpecialCharTok{\textasciitilde{}} \DecValTok{3}\NormalTok{,}
\NormalTok{    ALTER }\SpecialCharTok{\textless{}=} \DecValTok{85} \SpecialCharTok{\textasciitilde{}} \DecValTok{4}\NormalTok{,}
\NormalTok{    ALTER }\SpecialCharTok{\textgreater{}} \DecValTok{85} \SpecialCharTok{\textasciitilde{}} \DecValTok{5}
\NormalTok{  ))  }\SpecialCharTok{|\textgreater{}} 
\NormalTok{  sjlabelled}\SpecialCharTok{::}\FunctionTok{set\_labels}\NormalTok{(ALTER\_gr, }\AttributeTok{labels =} \FunctionTok{c}\NormalTok{(}\StringTok{"bis 25"}  \OtherTok{=} \DecValTok{1}\NormalTok{, }\StringTok{"26{-}45"} \OtherTok{=} \DecValTok{2}\NormalTok{, }\StringTok{"46{-}65"} \OtherTok{=} \DecValTok{3}\NormalTok{, }\StringTok{"66{-}85"} \OtherTok{=} \DecValTok{4}\NormalTok{, }\StringTok{"85+"} \OtherTok{=} \DecValTok{5}\NormalTok{)) }\SpecialCharTok{|\textgreater{}} 
\NormalTok{  sjlabelled}\SpecialCharTok{::}\FunctionTok{var\_labels}\NormalTok{(}\AttributeTok{ALTER\_gr =} \StringTok{"Alter gruppiert"}\NormalTok{)}

\NormalTok{DATEN\_BF }\SpecialCharTok{|\textgreater{}} \FunctionTok{frq}\NormalTok{(ALTER\_gr)}

\CommentTok{\# Hier werden für die Inhaltsanalyse die Anzahl Worte in Gruppen zusammengefasst.}
\NormalTok{DATEN\_IA }\OtherTok{\textless{}{-}}\NormalTok{ DATEN\_IA }\SpecialCharTok{|\textgreater{}} 
  \FunctionTok{mutate}\NormalTok{(}\AttributeTok{WORTE\_gr =} \FunctionTok{rec}\NormalTok{(WORT\_AZ, }
                        \AttributeTok{rec =} \StringTok{"0:100 = 1 [Kurzmeldung];}
\StringTok{                        101:150 = 2 [kurzer Artikel]; }
\StringTok{                        151:300 = 3 [länger]; }
\StringTok{                        301:500 = 4 [lang]; }
\StringTok{                        501:max = 5 [Langtext]"}\NormalTok{)) }\SpecialCharTok{|\textgreater{}} 
    \FunctionTok{relocate}\NormalTok{(WORTE\_gr, }\AttributeTok{.after =}\NormalTok{ WORT\_AZ) }


\DocumentationTok{\#\#\# Rekodierung der Anzahl Nennungen in wenige Gruppen}
\NormalTok{DATEN\_IA }\OtherTok{\textless{}{-}}\NormalTok{ DATEN\_IA }\SpecialCharTok{|\textgreater{}} 
  \FunctionTok{mutate}\NormalTok{(}\AttributeTok{Nenn\_Gr =} \FunctionTok{rec}\NormalTok{(Nenn\_Gesamt, }\AttributeTok{rec =} \StringTok{"1:3 = 1; 4:16 = 2; 17:100 = 3"}\NormalTok{)) }\SpecialCharTok{|\textgreater{}} 
\NormalTok{  sjlabelled}\SpecialCharTok{::}\FunctionTok{set\_labels}\NormalTok{(Nenn\_Gr, }\AttributeTok{labels =}\NormalTok{ (}\FunctionTok{c}\NormalTok{(}\StringTok{"selten"} \OtherTok{=} \DecValTok{1}\NormalTok{, }\StringTok{"mittel"} \OtherTok{=} \DecValTok{2}\NormalTok{, }\StringTok{"häufig"} \OtherTok{=} \DecValTok{3}\NormalTok{)))}
\end{Highlighting}
\end{Shaded}

Medium umkodieren

\begin{Shaded}
\begin{Highlighting}[]
\CommentTok{\# 1 SRF}
\CommentTok{\# 2 20min}
\CommentTok{\# 3 Watson}
\CommentTok{\# 4 NZZ}
\CommentTok{\# 5 Tagesanzeiger}
\CommentTok{\# 6 Blick}

\NormalTok{DATEN\_IA }\OtherTok{\textless{}{-}}\NormalTok{ DATEN\_IA }\SpecialCharTok{|\textgreater{}} 
  \FunctionTok{mutate}\NormalTok{(}\AttributeTok{Medium\_Q =} \FunctionTok{rec}\NormalTok{(MEDIUM, }
                    \AttributeTok{rec =} \StringTok{"1,4,5 = 1 [Qualität]; 2,3,6 = 2 [Boulevard]"}\NormalTok{)) }

\NormalTok{DATEN\_IA }\SpecialCharTok{|\textgreater{}}\NormalTok{ sjmisc}\SpecialCharTok{::}\FunctionTok{frq}\NormalTok{(Medium\_Q)}
\end{Highlighting}
\end{Shaded}

\hypertarget{bedingt-umkodieren}{%
\subsection{Bedingt umkodieren}\label{bedingt-umkodieren}}

\begin{Shaded}
\begin{Highlighting}[]
\NormalTok{DATEN\_BF }\SpecialCharTok{|\textgreater{}} 
\NormalTok{  sjmisc}\SpecialCharTok{::}\FunctionTok{frq}\NormalTok{(ALTER, BILDUNG\_gr)}

\NormalTok{DATEN\_BF }\OtherTok{\textless{}{-}}\NormalTok{ DATEN\_BF }\SpecialCharTok{|\textgreater{}} 
  \FunctionTok{mutate}\NormalTok{(}\AttributeTok{Alt\_Bild1 =} \FunctionTok{if\_else}\NormalTok{(ALTER }\SpecialCharTok{\textless{}} \DecValTok{35} \SpecialCharTok{\&}\NormalTok{ BILDUNG\_gr }\SpecialCharTok{==} \DecValTok{3}\NormalTok{, }\DecValTok{1}\NormalTok{, }\DecValTok{2}\NormalTok{)) }

\NormalTok{DATEN\_BF }\SpecialCharTok{|\textgreater{}}  \FunctionTok{frq}\NormalTok{(Alt\_Bild1)}

\CommentTok{\# Hier erstmal umkodieren uns anschauen}

\NormalTok{DATEN\_BF }\SpecialCharTok{|\textgreater{}} 
  \FunctionTok{mutate}\NormalTok{(}\AttributeTok{Alt\_Bild2 =} \FunctionTok{case\_when}\NormalTok{(}
\NormalTok{    ALTER }\SpecialCharTok{\textless{}} \DecValTok{35} \SpecialCharTok{\&}\NormalTok{ BILDUNG\_gr }\SpecialCharTok{==} \DecValTok{3} \SpecialCharTok{\textasciitilde{}} \DecValTok{1}\NormalTok{, }
\NormalTok{    ALTER }\SpecialCharTok{\textgreater{}} \DecValTok{75} \SpecialCharTok{\&}\NormalTok{ BILDUNG\_gr }\SpecialCharTok{==} \DecValTok{3} \SpecialCharTok{\textasciitilde{}} \DecValTok{2}\NormalTok{,}
\NormalTok{  )) }\SpecialCharTok{|\textgreater{}} 
  \FunctionTok{var\_labels}\NormalTok{(}\AttributeTok{Alt\_Bild2 =} \StringTok{"Alter und Bildung"}\NormalTok{) }\SpecialCharTok{|\textgreater{}} 
  \FunctionTok{set\_labels}\NormalTok{(Alt\_Bild2, }\AttributeTok{labels =} \FunctionTok{c}\NormalTok{(}\StringTok{"Junggebildet"} \OtherTok{=} \DecValTok{1}\NormalTok{, }\StringTok{"Altgebildet"} \OtherTok{=} \DecValTok{2}\NormalTok{)) }\SpecialCharTok{|\textgreater{}} 
  \FunctionTok{frq}\NormalTok{(Alt\_Bild2) }\CommentTok{\# erstmal nur eine Frequencies als Ergebnis und nicht gleich den Daten hinzufügen}

\CommentTok{\# Dann dem Datensatz hinzufügen}
\NormalTok{DATEN\_BF }\OtherTok{\textless{}{-}}\NormalTok{ DATEN\_BF }\SpecialCharTok{|\textgreater{}} 
  \FunctionTok{mutate}\NormalTok{(}\AttributeTok{Alt\_Bild2 =} \FunctionTok{case\_when}\NormalTok{(}
\NormalTok{    ALTER }\SpecialCharTok{\textless{}} \DecValTok{35} \SpecialCharTok{\&}\NormalTok{ BILDUNG\_gr }\SpecialCharTok{==} \DecValTok{3} \SpecialCharTok{\textasciitilde{}} \DecValTok{1}\NormalTok{, }
\NormalTok{    ALTER }\SpecialCharTok{\textgreater{}} \DecValTok{75} \SpecialCharTok{\&}\NormalTok{ BILDUNG\_gr }\SpecialCharTok{==} \DecValTok{3} \SpecialCharTok{\textasciitilde{}} \DecValTok{2}\NormalTok{,}
\NormalTok{  )) }\SpecialCharTok{|\textgreater{}} 
\NormalTok{  sjlabelled}\SpecialCharTok{::}\FunctionTok{var\_labels}\NormalTok{(}\AttributeTok{Alt\_Bild2 =} \StringTok{"Alter und Bildung"}\NormalTok{) }\SpecialCharTok{|\textgreater{}} 
\NormalTok{  sjlabelled}\SpecialCharTok{::}\FunctionTok{set\_labels}\NormalTok{(Alt\_Bild2, }\AttributeTok{labels =} \FunctionTok{c}\NormalTok{(}\StringTok{"Junggebildet"} \OtherTok{=} \DecValTok{1}\NormalTok{, }\StringTok{"Altgebildet"} \OtherTok{=} \DecValTok{2}\NormalTok{))}
\end{Highlighting}
\end{Shaded}

\hypertarget{variablen-zu-indizes-zusammenfassen}{%
\subsection{Variablen zu Indizes
zusammenfassen}\label{variablen-zu-indizes-zusammenfassen}}

\begin{Shaded}
\begin{Highlighting}[]
\CommentTok{\# Ein Summenindex für 10 Variablen zur Nachrichtennutzung Es wird zeilenweise die Summe für die Dummys (0/1{-}Variablen) berechnet}

\NormalTok{DATEN\_BF }\OtherTok{\textless{}{-}}\NormalTok{ DATEN\_BF }\SpecialCharTok{|\textgreater{}} 
  \FunctionTok{mutate}\NormalTok{(}\AttributeTok{MEDIEN\_sum =} \FunctionTok{rowSums}\NormalTok{(}\FunctionTok{across}\NormalTok{(NACHRICHT\_TITEL1}\SpecialCharTok{:}\NormalTok{NACHRICHT\_TITEL10), }\AttributeTok{na.rm =} \ConstantTok{TRUE}\NormalTok{))}

\NormalTok{DATEN\_BF }\SpecialCharTok{|\textgreater{}} 
\NormalTok{  sjmisc}\SpecialCharTok{::}\FunctionTok{frq}\NormalTok{(MEDIEN\_sum)}

\CommentTok{\# Hier noch der Durchschnittswert als Index, was bei Dummys der Anteil der 1{-}er ist, also wenn 68 von knapp 200 Leuten 2 Medien nannten, dann gibt es 68 mal die 0.2 }

\NormalTok{DATEN\_BF }\OtherTok{\textless{}{-}}\NormalTok{ DATEN\_BF }\SpecialCharTok{|\textgreater{}} 
  \FunctionTok{mutate}\NormalTok{(}\AttributeTok{MEDIEN\_mean =} \FunctionTok{rowMeans}\NormalTok{(}\FunctionTok{across}\NormalTok{(NACHRICHT\_TITEL1}\SpecialCharTok{:}\NormalTok{NACHRICHT\_TITEL10), }\AttributeTok{na.rm =} \ConstantTok{TRUE}\NormalTok{))}

\NormalTok{DATEN\_BF }\SpecialCharTok{|\textgreater{}} 
\NormalTok{  sjmisc}\SpecialCharTok{::}\FunctionTok{frq}\NormalTok{(MEDIEN\_mean)}

\NormalTok{mutate}

\NormalTok{DATEN\_IA }\OtherTok{\textless{}{-}}\NormalTok{ DATEN\_IA }\SpecialCharTok{|\textgreater{}} 
  \FunctionTok{mutate}\NormalTok{(}\FunctionTok{across}\NormalTok{(}\FunctionTok{c}\NormalTok{(PERS}\SpecialCharTok{:}\NormalTok{KNTR, KAT}\SpecialCharTok{:}\NormalTok{NAEHE), }\SpecialCharTok{\textasciitilde{}}\FunctionTok{replace}\NormalTok{(.x, }\FunctionTok{is.na}\NormalTok{(.x),}\DecValTok{0}\NormalTok{))) }\SpecialCharTok{|\textgreater{}} 
  \FunctionTok{mutate}\NormalTok{(}\FunctionTok{across}\NormalTok{(}\FunctionTok{c}\NormalTok{(PERS}\SpecialCharTok{:}\NormalTok{KNTR, KAT}\SpecialCharTok{:}\NormalTok{NAEHE), }\SpecialCharTok{\textasciitilde{}}\FunctionTok{replace}\NormalTok{(.x, .x }\SpecialCharTok{==} \SpecialCharTok{{-}}\DecValTok{9}\NormalTok{,}\DecValTok{0}\NormalTok{))) }\SpecialCharTok{|\textgreater{}} 
  \FunctionTok{rowwise}\NormalTok{() }\SpecialCharTok{|\textgreater{}} 
  \FunctionTok{mutate}\NormalTok{(}\AttributeTok{Jour\_Beacht =} \FunctionTok{sum}\NormalTok{(}\FunctionTok{c}\NormalTok{(PERS}\SpecialCharTok{:}\NormalTok{KNTR, KAT}\SpecialCharTok{:}\NormalTok{NAEHE)))}

\NormalTok{DATEN\_IA }\SpecialCharTok{|\textgreater{}}\NormalTok{ sjmisc}\SpecialCharTok{::}\FunctionTok{frq}\NormalTok{(Jour\_Beacht)}
\end{Highlighting}
\end{Shaded}

\hypertarget{umfangreiche-umkodierungstabellen-tf-e}{%
\subsection{Umfangreiche Umkodierungstabellen (TF
E)}\label{umfangreiche-umkodierungstabellen-tf-e}}

Zum Beispiel für Ländereigenschaften über die Ländercodes

\begin{Shaded}
\begin{Highlighting}[]
\NormalTok{vembedr}\SpecialCharTok{::}\FunctionTok{embed\_youtube}\NormalTok{(}\StringTok{"5BR3m7E6E{-}s"}\NormalTok{) }\SpecialCharTok{|\textgreater{}} 
\NormalTok{  vembedr}\SpecialCharTok{::}\FunctionTok{use\_rounded}\NormalTok{()}
\end{Highlighting}
\end{Shaded}

\begin{Shaded}
\begin{Highlighting}[]
\CommentTok{\# Aus der Länderliste mit den Kodierungen kommen die Variablen für eine Laenderliste:}
\NormalTok{Laenderliste }\OtherTok{\textless{}{-}}\NormalTok{ readxl}\SpecialCharTok{::}\FunctionTok{read\_xlsx}\NormalTok{(}\StringTok{"Daten/Laender\_Liste\_Kodierung.xlsx"}\NormalTok{, }\AttributeTok{sheet =} \StringTok{"Länderliste"}\NormalTok{)}

\CommentTok{\# Die Codes für die Länder kommen aus einer anderen Untertabelle der Excel, darum steht hinten bei sheets etwas anderes:}
\NormalTok{Ländercodes }\OtherTok{\textless{}{-}}\NormalTok{ readxl}\SpecialCharTok{::}\FunctionTok{read\_xlsx}\NormalTok{(}\StringTok{"Daten/Laender\_Liste\_Kodierung.xlsx"}\NormalTok{, }\AttributeTok{sheet =} \StringTok{"Ländercodes"}\NormalTok{)}

\CommentTok{\# Dann gibt es noch lauter weitere Codes, die wir jetzt anmatchen können, die können wir später gut gebrauchen, weil verschiedene Datensätze aus dem Internetz verschiedene Ländercodes verwenden}

\NormalTok{Land\_AlphaISO }\OtherTok{\textless{}{-}}\NormalTok{ readxl}\SpecialCharTok{::}\FunctionTok{read\_xlsx}\NormalTok{(}\StringTok{"Daten/Laender\_Liste\_Kodierung.xlsx"}\NormalTok{, }\AttributeTok{sheet =} \StringTok{"AlphaISO"}\NormalTok{) }\SpecialCharTok{|\textgreater{}} 
  \FunctionTok{select}\NormalTok{(name, }\StringTok{"alpha2"}\NormalTok{, }\StringTok{"alpha3"}\NormalTok{, }\StringTok{"CountryCode"}\NormalTok{)}

\CommentTok{\# Hier binden wir die zusammen, erst die Ländercodes an die Laenderliste und dann noch die Land\_AlphaISO hinten dran}
\NormalTok{Laender }\OtherTok{\textless{}{-}} \FunctionTok{left\_join}\NormalTok{(Laenderliste, Ländercodes, }\AttributeTok{by =} \StringTok{"Land"}\NormalTok{) }\SpecialCharTok{|\textgreater{}} 
  \FunctionTok{left\_join}\NormalTok{(., Land\_AlphaISO, }\AttributeTok{by =} \StringTok{"alpha2"}\NormalTok{)}

\CommentTok{\# oh, und hier noch ein paar Handelsdaten, falls das jemanden interessiert}
\NormalTok{WorldTrade }\OtherTok{\textless{}{-}}\NormalTok{ economiccomplexity}\SpecialCharTok{::}\NormalTok{world\_gdp\_avg\_1998\_to\_2000}

\CommentTok{\# So, die kleben wir jetzt auch noch hinten an unsere Laendervariable}
\NormalTok{Laender }\OtherTok{\textless{}{-}} \FunctionTok{left\_join}\NormalTok{(Laender, WorldTrade, }\AttributeTok{by =} \FunctionTok{c}\NormalTok{(}\StringTok{"alpha3"} \OtherTok{=} \StringTok{"country"}\NormalTok{))}

\CommentTok{\# mit diesem Paket können Sie die Coronazahlen aller Länder als Datensatz laden, für jeden Tag}
\NormalTok{Covid\_JHU }\OtherTok{\textless{}{-}}\NormalTok{ tidycovid19}\SpecialCharTok{::}\FunctionTok{download\_jhu\_csse\_covid19\_data}\NormalTok{() }\CommentTok{\# das dauert bischen}
\NormalTok{Covid\_WBank }\OtherTok{\textless{}{-}}\NormalTok{ tidycovid19}\SpecialCharTok{::}\FunctionTok{download\_wbank\_data}\NormalTok{()}

\CommentTok{\# jetzt nehme ich den Covid{-}Datensatz und ...}
\NormalTok{Covid\_Land }\OtherTok{\textless{}{-}}\NormalTok{ Covid\_JHU }\SpecialCharTok{|\textgreater{}} 
  \FunctionTok{filter}\NormalTok{(date }\SpecialCharTok{\textgreater{}=} \StringTok{"2020{-}12{-}4"} \SpecialCharTok{\&}\NormalTok{ date }\SpecialCharTok{\textless{}=} \StringTok{"2020{-}12{-}6"}\NormalTok{) }\SpecialCharTok{|\textgreater{}} \CommentTok{\# filtere so, dass nur die Daten der Feldphase bleiben (auf die Tage der Feldphase achten)}
  \FunctionTok{group\_by}\NormalTok{(iso3c) }\SpecialCharTok{|\textgreater{}} \CommentTok{\# hier gruppiere ich nach den Ländern, die im Code iso3c (dreistellig character) stehn}
  \FunctionTok{summarise}\NormalTok{(}\AttributeTok{C19\_Gesamt =} \FunctionTok{max}\NormalTok{(confirmed), }\CommentTok{\# hier summiert R die maximal bestätigten (also kummulierten) Fallzahlen}
            \AttributeTok{C19\_Feldwoche\_krank =} \FunctionTok{sum}\NormalTok{(confirmed }\SpecialCharTok{{-}}\NormalTok{ recovered)) }\CommentTok{\# hier noch die Summe der Kranken in der Feldphase}

\CommentTok{\# Hier kleben wir wieder die Covid{-}Daten an die Laender dran}
\NormalTok{Laender }\OtherTok{\textless{}{-}} \FunctionTok{left\_join}\NormalTok{(Laender, Covid\_Land, }\AttributeTok{by =} \FunctionTok{c}\NormalTok{(}\StringTok{"alpha3"} \OtherTok{=} \StringTok{"iso3c"}\NormalTok{)) }

\CommentTok{\# Schreibe die Laender{-}Daten raus in die Exceldatei "Laender\_codes.xlsx"}
\NormalTok{writexl}\SpecialCharTok{::}\FunctionTok{write\_xlsx}\NormalTok{(Laender, }\StringTok{"Daten/Laender\_codes.xlsx"}\NormalTok{)}

\CommentTok{\# Hier die Laender{-}Daten wieder aus der (eventuell ergänzten Laender{-}Excel) in R laden:}
\NormalTok{Laender }\OtherTok{\textless{}{-}}\NormalTok{ readxl}\SpecialCharTok{::}\FunctionTok{read\_xlsx}\NormalTok{(}\StringTok{"Daten/Laender\_codes.xlsx"}\NormalTok{)}

\CommentTok{\# Jetzt kommt die Magie. Wie matchen die Laendervariable an die Inhaltsanalysedaten:}

\CommentTok{\# Hier lade ich die Inhaltanalysedaten aus meiner Excel. Vermutlich heisst die Excel bei Ihnen anders und liegt vielleicht auch nicht im Unterordner "Daten".}
\NormalTok{IA }\OtherTok{\textless{}{-}}\NormalTok{ readxl}\SpecialCharTok{::}\FunctionTok{read.xlsx}\NormalTok{(}\StringTok{"Daten/IA\_Gesamt.xlsx"}\NormalTok{)}

\CommentTok{\# Hier werden alle Variablen aus der Laendertabelle hinten an die Inhaltsanalyse geklebt}
\NormalTok{IA }\OtherTok{\textless{}{-}} \FunctionTok{left\_join}\NormalTok{(IA, Laender, }\AttributeTok{by =} \FunctionTok{c}\NormalTok{(}\StringTok{"LAND"} \OtherTok{=} \StringTok{"Code"}\NormalTok{))}
\end{Highlighting}
\end{Shaded}

\bookmarksetup{startatroot}

\hypertarget{forschungsbericht-mit-quartoword}{%
\chapter{Forschungsbericht mit
Quarto/Word}\label{forschungsbericht-mit-quartoword}}

In der modernen quantitativen Forschung arbeiten die meisten Kollegen
mit R (oder Python, aber die sind komisch). Mehr und mehr schreiben ihre
Forschungsberichte in R-Markdown beziehungsweise neuerdings Quarto, was
(fast) dasselbe ist. Wenn sie das tun, dann können sie Ihre Arbeit
später als Word-Datei rauslassen, als HTML (um zum Beispiel eine
Onlinedokumentation oder Onlineanhänge zu Ihren Arbeiten
bereitzustellen) und als PDF um sie verteilen oder auch drucken zu
können. Die PDF basiert auf einem der ausgereiftesten Drucksatzsystem,
das auch noch kostenlos ist: LaTeX. Dieses Drucksatzsytem ist
kompliziert, aber Markdown nicht. Markdown wurde entwickelt, damit man
Webseiten einfach erstellen kann und die Texte auch im Rohzustand noch
lesbar sind. Darum ist es viel einfacher Markdown zu verwenden als Word,
HTML, LaTeX oder gleich PostScript. Quarto ist eine Weiterentwicklung
von Markdown für wissenschaftliche Texte. Dieses System ermöglicht Open
Source Science und Publishing.

Wissenschaftliches Arbeiten bedeutet auch immer zitieren. Die Zitationen
können natürlich irgendwie händisch eingebaut und später in einem
Literaturverzeichnis ausgeführt werden. Das ist allerdings 1. aufwendig,
2. fehleranfällig und 3. sehr unflexibel (weil man zum Beispiel alles
ändern müsste, statt nur einen Zitationsstil anzupassen.) Für Zitationen
in RMarkdown (und Quarto) kann das Format BibTeX verwendet werden. Das
ist schon vor langer Zeit für TeX und LaTeX entwickelt worden und ist
sehr verbreitet (sie Finden fast immer für Zitationen die sogenannten
Bib-Tex-Schlüssel). Ein kostenloser Editor, für den es für alle Systeme
und Online ein Frontend gibt ist Zotero (Citavi ist auch ok, aber nicht
überall kostenlos und darum eigentlich nicht so verbreitet). Dort können
Sie einfach ISBN-Nummern oder auch DOIs eingeben, um sämtliche
Zitationsangaben abrufen zu können. Sie müssen die also nicht selbst
eingeben (können aber auch das tun). Zotero lässt sich gut in RStudio
integrieren (Tools -\textgreater{} Project Options -\textgreater{} R
Markdown -\textgreater{} Visual Mode Zotero).
\href{https://www.youtube.com/watch?v=PqQp_oUUY5w}{Hier gibt es eine
Einführung in Zotero, die zwar mehr auf Word gemünzt ist, aber gut.} Im
Youtube-Video unten von Ulrik Lyngs wiederum finden Sie eine komplette
Zusammenfassung unseres ganzen Workflows mal von einem anderen Typen.
Sie erkennen schnell, dass das Setup, das ich Ihnen ans Herz legen
möchte, nicht nur eine absonderliche Marotte meinerseits ist, sondern
State of the Art wissenschaftlichen Publizierens, das Sie ja im Studium
lernen sollten. :-)

\begin{Shaded}
\begin{Highlighting}[]
\NormalTok{vembedr}\SpecialCharTok{::}\FunctionTok{embed\_youtube}\NormalTok{(}\StringTok{"aU0uetWJXis"}\NormalTok{) }\SpecialCharTok{|\textgreater{}} 
\NormalTok{  vembedr}\SpecialCharTok{::}\FunctionTok{use\_rounded}\NormalTok{()}
\end{Highlighting}
\end{Shaded}

\hypertarget{quarto-ist-besser-und-einfacher}{%
\section{Quarto ist besser und
einfacher}\label{quarto-ist-besser-und-einfacher}}

Zu Quarto zwinge ich Sie nicht, aber die hier gezeigte Methode zur
Erstellung von Hausarbeiten, Forschungsberichten und sonstigen Texten
ist schon sehr leistungsstark, weil Sie nicht nur tolle Berichte in PDF
produzieren können, sondern Ihren Bericht auch gleich in HTML rauslassen
können oder auch als ebook oder auch als Word (mit Einschränkungen).
Wenn Sie Ihren Bericht in Quarto verfassen, dann bekommen Sie am Ende
des Semesters mit Ihren Task-Force-Zeugnissen eine Bestätigung, dass Sie
auch dieses Tool beherrschen. Quarto-Markdown zu beherrschen ist eine
Qualifikation, die Sie sich gut in Ihr CV schreiben können. Auf dieser
Seite \url{https://quarto.org} finden Sie einige Bücher zu bookdown, die
alle auch in bookdwon verfasst wurden. Hier \url{https://quarto.org}
finden Sie Lösungen zu den meisten Fragen zu Quarto.

Diese R-Introduction ist übrigens auch in bookdown verfasst.

\begin{Shaded}
\begin{Highlighting}[]
\NormalTok{vembedr}\SpecialCharTok{::}\FunctionTok{embed\_youtube}\NormalTok{(}\StringTok{"7\_WjjClQGUw"}\NormalTok{) }\SpecialCharTok{|\textgreater{}} 
\NormalTok{  vembedr}\SpecialCharTok{::}\FunctionTok{use\_rounded}\NormalTok{()}
\end{Highlighting}
\end{Shaded}

\hypertarget{tabellen-und-grafiken-nach-word-schaffen}{%
\section{Tabellen und Grafiken nach Word
schaffen}\label{tabellen-und-grafiken-nach-word-schaffen}}

Es ist eine Herausforderung, Berichte in Markdown bzw. bookdown zu
setzen und (wie alles in R) braucht es viel Zeit. Die Alternative
besteht in vertrauteren Textverarbeitungsprogrammen wie zB Word, Page
oder Open-Office. Berichte mit Titelblatt, Inhalts-, Abbildungs- sowie
Tabellenverzeichnis, automatisch beschrifteten Grafiken und Tabellen
sowie Querverweisen und einem ordentlichen Literaturverzeichnis
hinzubekommen ist allerdings auch nicht einfach, wenn man diese
Textverarbeitungsprogramme benutzt. Ganz zu schweigen von schwer
kontrollierbaren Platzierungen von Grafiken und Tabellen. Die fliegen
einem in solchen Programmen schnell um die Ohren.

Alle Bedenken beiseite, gab es den Wunsch, dass auch gezeigt wird, wie
man die Grafiken aus R in hinreichender Qualität nach Word rüberschafft
(in anderen Programmen funktioniert es vergleichbar; Word nehme ich hier
als das gängigste Beispiel). Screenshots weisen keine akzeptable
Qualität auf und sind deshalb keine Lösung -- vor allem nicht für
Tabellen.

Tabellen können schlicht in Wordtabellen abgeschrieben werden, was
allerdings zu Übertragungsfehlern führt. Die Tabellen können aber auch
schlicht kopiert und nach Word übertragen werden. Dort müssen Sie
allerdings noch in eine annehmbare Form gebracht werden, was mit
Tabellenformatierung oder Tabellenvorlagen einigermassen funktioniert.
Die Tabellenüberschriften werden mit \enquote{Beschriftung einfügen}
gemacht.

Noch komplizierter ist die Übertragung von Diagrammen und anderen
Bildformaten aus R nach Word. Eine Möglichkeit bietet \texttt{ggplot2},
da bei ggplot-Grafiken mit \texttt{ggsave} die Grafiken im pdf-Format
oder als png gespeichert werden können (wo möglich, ist pdf deutlich
besser). Alternativ kann der \enquote{Visual} Mode im RStudio-Editor
verwendet und dort Grafiken per drag-and-drop direkt nach Word
rübergezogen werden. Die verschiedenen Methoden zeige ich hier im Video.

\begin{Shaded}
\begin{Highlighting}[]
\NormalTok{vembedr}\SpecialCharTok{::}\FunctionTok{embed\_youtube}\NormalTok{(}\StringTok{"0gy2bIfJpnk"}\NormalTok{) }\SpecialCharTok{|\textgreater{}} 
\NormalTok{  vembedr}\SpecialCharTok{::}\FunctionTok{use\_rounded}\NormalTok{()}
\end{Highlighting}
\end{Shaded}


%%%%%%%%%%%%%%%%%%%%%%%%%%%%%%%%%%

             %Anhang

%%%%%%%%%%%%%%%%%%%%%%%%%%%%%%%%%%


\end{document}
